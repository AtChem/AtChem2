The \texttt{tools/} directory contains a number of auxiliary scripts to
install, build and compile AtChem2, and to plot the results of the
model:

\begin{itemize}
\tightlist
\item
  shell script to compile the model: \texttt{build.sh}.
\item
  Python scripts to process the chemical mechanism:
  \texttt{fix\_mechanism\_fac.py}, \texttt{mech\_converter.py}.
\item
  Python scripts to enforce a consistent {[}{[}coding style\textbar{}3.2
  Style Guide{]}{]}: \texttt{fix\_indent.py}, \texttt{fix\_style.py}.
\item
  Ruby script to run the unit tests: \texttt{fruit\_generator.rb}.
\item
  example chemical mechanism in FACSIMILE format:
  \texttt{mcm\_example.fac}.
\item
  \texttt{install/} directory containing scripts to install the
  {[}{[}dependencies\textbar{}1.1 Dependencies{]}{]}.
\item
  \texttt{plot/} directory containing scripts to plot the model results.
\end{itemize}

In addition, the \texttt{tools/} directory contains a copy of the
\texttt{Makefile}, which has to be copied to the \emph{main directory}
and modified as explained in the {[}{[}installation page\textbar{}1.
Installation{]}{]}.

\hypertarget{plot-tools}{%
\subsection{Plot tools}\label{plot-tools}}

The plotting scripts in \texttt{tools/plot/} are only intended to give a
quick view of the model results. It is suggested to use a proper data
analysis software (e.g., R, Octave/MATLAB, Igor, Origin, etc\ldots{}) to
process and analyze the model results. The scripts are written in
various programming languages, but they all produce the same output: a
file called \texttt{atchem2\_output.pdf} in the given directory (e.g.,
\texttt{model/output/}).

From the \emph{main directory}:

\begin{verbatim}
gnuplot -c tools/plot/plot-atchem2.gp model/output/
octave tools/plot/plot-atchem2.m model/output/
python tools/plot/plot-atchem2.py model/output/
Rscript --vanilla tools/plot/plot-atchem2.r model/output/
\end{verbatim}

\emph{N.B.}: the matlab script (\texttt{plot-atchem2.m}) is compatible
with both Octave and MATLAB. GNU Octave is an open-source implementation
of MATLAB.
