\section{Model Parameters} \label{sec:parameters}

The \textbf{model parameters} are set in the text file
\texttt{model/configuration/model.parameters}; they control the general
setup of the model.

\begin{itemize}
\item
  \textbf{number of steps} and \textbf{step size}. The duration of the
  model run is determined by the number of steps and the step size (in
  seconds). The step size controls the frequency of the model output for
  the chemical species listed in \texttt{outputSpecies.config} (see
  {[}{[}2.6 Config Files{]}{]}), and for the environment variables, the
  photolysis rates, the diagnostic variables.\\
  For example, a model runtime of 2 hours, with output every 5 minutes,
  requires 24 steps with a step size of 300 seconds (24x300 = 7200 sec =
  2 hours). Possible values for these parameters are shown below, for
  reference.
\item
  \textbf{species interpolation method} and \textbf{conditions
  interpolation method}. Interpolation method used for the constrained
  chemical species, and for the constrained environment variables and
  the photolysis rates, respectively (see {[}{[}2.7 Constraints{]}{]}).
  Two interpolation methods are currently implemented in AtChem2:
  piecewise constant (\texttt{1}) and piecewise linear (\texttt{2}). The
  default option is \emph{piecewise linear interpolation}.
\item
  \textbf{rates output step size}. Frequency (in seconds) of the model
  output for the production and loss rates of selected species. The
  species for which this parameter is required are listed in
  \texttt{outputRates.config} (see {[}{[}2.6 Config Files{]}{]}).
\item
  \textbf{model start time}. Start time of the model (in seconds)
  calculated from midnight of the \textbf{day}, \textbf{month},
  \textbf{year} parameters (see below). For example, a start time of
  3600 means the model run starts at 1:00 in the morning and a start
  time of 43200 means the model run starts at midday. The \textbf{model
  stop time} is automatically calculated as:
  \texttt{model\ start\ time\ +\ (number\ of\ steps\ *\ step\ size))}.\\
  \emph{Important}: if one or more variables are constrained, the
  interval between the model start time and the model stop time must be
  equal or shorter than the time interval of the constrained data (see
  {[}{[}2.7 Constraints{]}{]}).
\item
  \textbf{jacobian output step size}. Frequency of the model output for
  the Jacobian matrix (in seconds). If the frequency is set to
  \texttt{0} (default option), the Jacobian matrix is not output. Note
  that the \texttt{jacobian.output} file generated by the model can be
  very large, especially if the chemical mechanism has many reactions
  and/or the model runtime is long.
\item
  \textbf{latitude} and \textbf{longitude}. Geographical coordinates (in
  degrees). By convention, latitude North is positive and latitude South
  is negative, longitude East is negative and longitude West is
  positive. Latitude and longitude are used only for the calculation of
  the Earth-Sun angles, which are needed for the MCM photolysis
  parameterisation (see {[}{[}2.5 Photolysis Rates and JFAC{]}{]}).
\item
  \textbf{day} and \textbf{month} and \textbf{year}. Start date of the
  model simulation. The model time is in seconds since midnight of the
  start date.
\item
  \textbf{reaction rates output step size}. Frequency (in seconds) of
  the model output for the reaction rates of every reaction in the
  chemical mechanism. The reaction rates are saved in the directory
  \texttt{model/output/reactionRates/} as one file for each model step,
  with the name of the file corresponding to the time in seconds. In
  previous versions of AtChem, this output was called
  \emph{instantaneous rates}.\\
  Note that this parameter is different from \textbf{rates output step
  size} (see above), which sets the frequency of a formatted output of
  reaction rates for selected species of interest. For more information
  go to: {[}{[}2.6 Config Files{]}{]}.
\end{itemize}

\begin{center}\rule{0.5\linewidth}{\linethickness}\end{center}

\hypertarget{runtime-reference-values}{%
\subsection{Runtime reference values}\label{runtime-reference-values}}

For 1 day at 15 minute intervals:

\begin{verbatim}
96      number of steps
900     step size
\end{verbatim}

For 2 days at 15 minute intervals:

\begin{verbatim}
192     number of steps
900     step size
\end{verbatim}

For 2 days at 1 minute intervals:

\begin{verbatim}
2880    number of steps
60      step size
\end{verbatim}
