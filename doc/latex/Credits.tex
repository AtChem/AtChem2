% -----------------------------------------------------------------------------
%
% Copyright (c) 2017 Sam Cox, Roberto Sommariva
%
% This file is part of the AtChem2 software package.
%
% This file is covered by the MIT license which can be found in the file
% LICENSE.md at the top level of the AtChem2 distribution.
%
% -----------------------------------------------------------------------------

\chapter{Credits, Acknowledgements \& Funding} \label{ch:credits}

\setlength{\parindent}{0pt}

% -------------------------------------------------------------------- %
\section{Credits} \label{sec:credits}

AtChem2 has been developed at the University of Leicester by:

\begin{itemize}
\item Sam Cox
\item Roberto Sommariva (also at the University of Birmingham)
\end{itemize}

Additional code has been contributed by (in alphabetical order):

\begin{itemize}
\item James Allsopp
\item Will Drysdale
\item Maarten Fabr{\'e}
\item Alfred Mayhew
\item Killian Murphy
\item Beth Nelson
\item Mike Newland
\item Marios Panagi
\end{itemize}

AtChem2 is a development of \href{https://atchem.york.ac.uk}{AtChem-online},
which was created at the University of Leeds (and is now hosted at the
University of York) by:

\begin{itemize}
\item Chris Martin
\item Kasia Boro{\'n}ska
\item Jenny Young
\item Peter Jimack
\item Mike Pilling
\end{itemize}

Model evaluation and testing of AtChem-online was done by Andrew
Rickard (NCAS/University of York) and Monica V{\'a}zquez-Moreno
(CEAM/EUPHORE), and technical support was provided by David Waller
(University of Leeds).

% ------------------------------------------------------------------- %
\section{Acknowledgements} \label{sec:acknowledgements}

Thanks for their support, feedback, and contributions to (in
alphabetical order):

\begin{itemize}
\item Bill Bloss
\item Peter Br{\"a}uer
\item Nahid Chowdhury
\item Stuart Lacy
\item Vasilis Matthaios
\item Paul Monks
\item Jon Wakelin
\item Rob Woodward-Massey
\end{itemize}

Many thanks to Harald Stark (University of Colorado-Boulder, USA) for
providing the observational data used to test the photolysis rates
subroutines.

% -------------------------------------------------------------------- %
\section{Funding} \label{sec:funding}

Funding has been provided, at different stages, by:

\begin{itemize}
\item EUROCHAMP project: \url{https://www.eurochamp.org}.
\item MetOffice: \url{https://www.metoffice.gov.uk}.
\item National Centre for Atmospheric Science (NCAS):\\ \url{https://www.ncas.ac.uk}.
\item Natural Environment Research Council (NERC):\\ \url{https://nerc.ukri.org}.
\item University of Birmingham, Research Software Group.
\item University of Leicester, ReSET programme.
\end{itemize}
