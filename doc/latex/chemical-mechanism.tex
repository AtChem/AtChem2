\section{Chemical Mechanism} \label{sec:mechanism}

The \textbf{chemical mechanism} is the core element of an atmospheric
chemistry box-model. In AtChem2, the mechanism file is written in
FACSIMILE format and has the extension \texttt{.fac}. The FACSIMILE
format is used to describe chemical reactions in the commercial
\href{http://www.mcpa-software.com/}{FACSIMILE Kinetic Modelling
Software}; for historical reasons, the software and the format have been
widely used in conjunction with the MCM. The
\href{http://mcm.leeds.ac.uk/MCMv3.3.1/extract.htt}{extraction tool} on
the MCM website can generate \texttt{.fac} files directly in FACSIMILE
format.

\hypertarget{facsimile-format}{%
\subsection{FACSIMILE format}\label{facsimile-format}}

Chemical reactions are described in FACSIMILE format using the following
notation:

\begin{verbatim}
% k : A + B = C + D ;
\end{verbatim}

where \texttt{k} is the rate coefficient, \texttt{A} and \texttt{B} are
the reactants, \texttt{C} and \texttt{D} are the products. The reaction
starts with the \texttt{\%} character and ends with the \texttt{;}
character. Comments - enclosed between the characters \texttt{*} and
\texttt{;} - can be added to the mechanism, and will be ignored by the
compiler. For example:

\begin{verbatim}
* conversion of A to C with rate coefficient of 1e-4 *;
% 1E-4 : A = C ;
\end{verbatim}

The mechanism file is processed by the script \texttt{tools/build.sh},
as explained in the {[}{[}model setup page\textbar{}2. Model Setup and
Execution{]}{]}. For the build process to work, the \texttt{.fac} file
must include four sections delimited by a single comment line, which
allows the script to recognize the beginning of each section: 1. Generic
rate coefficients 1. Complex reactions rate coefficients 1. Sum of
peroxy radicals (see below) 1. Chemical Reactions

These comment lines must always be present, even though the respective
sections can be empty. A minimal \texttt{.fac} file looks like this:

\begin{verbatim}
* Generic Rate Coefficients ;

* Complex reactions ;

* Peroxy radicals. ;

RO2 =  ;

* Reaction definitions. ;

% k : A + B = C ;
\end{verbatim}

A simple chemical mechanism in FACSIMILE format - with the first step of
the atmospheric oxidation of ethanol - is shown below, as an example.

\hypertarget{ro2-sum}{%
\subsection{RO2 sum}\label{ro2-sum}}

The sum of organic peroxy radicals (RO2) is a key component of the
Master Chemical Mechanism (see the MCM protocol papers:
\href{https://doi.org/10.1016/S1352-2310(96)00105-7}{Jenkin et al.,
Atmos. Environ., 31, 81-104, 1997} and
\href{https://doi.org/10.5194/acp-3-161-2003}{Saunders et al., Atmos.
Chem. Phys., 3, 161-180, 2003}). Since AtChem2 is designed primarily to
run models based upon the MCM, the \texttt{.fac} file must contain a
section with the RO2 sum. This section must be introduced by the comment
line \texttt{*\ Peroxy\ radicals.\ ;} (see above) and has the format:

\begin{verbatim}
RO2 = RO2a + RO2b + RO2c + ... ;
\end{verbatim}

where \texttt{RO2a}, \texttt{RO2b}, \texttt{RO2c}, are the organic
peroxy radicals in the chemical mechanism. If there are no organic
peroxy radicals in the mechanism (or if the mechanism is not based upon
the MCM), the RO2 sum must be left empty, e.g.:

\begin{verbatim}
RO2 = ;
\end{verbatim}

\emph{Important}: HO2 is a peroxy radical, but it is not an organic
molecule. Therefore it should NOT be included in the RO2 sum.

The RO2 sum is automatically generated from the mechanism file during
the build process, using the list of RO2 extracted from the MCM
database. AtChem2 includes the list of all the organic peroxy radicals
in version 3.3.1 of the MCM (\texttt{mcm/peroxy-radicals\_v3.3.1}),
which is used by default. Since v1.1, lists of organic peroxy radicals
from other versions of the MCM are also included in the \texttt{mcm/}
directory: see the file \texttt{mcm/INFO.md} for instructions on how to
use previous versions of the MCM with AtChem2.

\hypertarget{the-mcm-extraction-tool}{%
\subsection{The MCM extraction tool}\label{the-mcm-extraction-tool}}

The MCM website provides a convenient tool which can be used to download
the whole MCM, or subsets of it, in FACSIMILE format. After selecting
the species of interest in the
\href{http://mcm.leeds.ac.uk/MCMv3.3.1/roots.htt}{MCM browser}, add them
to the \emph{Mark List}, then proceed to the
\href{http://mcm.leeds.ac.uk/MCMv3.3.1/extract.htt}{MCM extraction tool}
and select \emph{FACSIMILE} as format. Make sure to tick the boxes:

\begin{verbatim}
[x] Include inorganic reactions?
[x] Include generic rate coefficients? FACSIMILE, FORTRAN and KPP formats only
\end{verbatim}

then press the \emph{Extract} button to download the generated
\texttt{.fac} file into a directory of choice (e.g., \texttt{model/};
see the {[}{[}model structure page\textbar{}1.2 Model Structure{]}{]}).
The mechanism can be modified with a text editor (if necessary) or
directly used in AtChem2. More information about the MCM browser and the
extractor tool can be found on the \href{http://mcm.leeds.ac.uk}{MCM
website}.

\hypertarget{the-build-process}{%
\subsection{The build process}\label{the-build-process}}

Atchem2 uses a Python script (\texttt{tools/mech\_converter.py},
automatically called by \texttt{tools/build.sh} during the build
process) to convert the chemical mechanism into a format that can be
read by the Fortran code.

The script generates one Fortran file, one shared library, and four
configuration files from the \texttt{*.fac} file:

\begin{itemize}
\item
  \textbf{mechanism.f90} contains the equations, in Fortran code, to
  calculate the rate coefficients of each reaction. By default, it is
  placed in \texttt{model/configuration/}.
\item
  \textbf{mechanism.so} is the compiled version of
  \texttt{mechanism.f90}. By default, it is placed in
  \texttt{model/configuration/}.
\item
  \textbf{mechanism.species} contains the list of chemical species in
  the mechanism. By default, it is saved in
  \texttt{model/configuration/}. The file has no header. The first
  column is the \emph{ID number} of each species, the second column is
  the name of the species:
  \texttt{1\ O\ \ \ 2\ O3\ \ \ 3\ NO\ \ \ 4\ NO2}
\item
  \textbf{mechanism.reac} and \textbf{mechanism.prod} contain the
  reactants and the products (respectively) in each reaction of the
  mechanism. By default, it is saved in \texttt{model/configuration/}.
  The files have a 1 line header with the number of species, the number
  of reactions and the number of equations in the Generic Rate
  Coefficients and Complex Reactions sections. The first column is the
  \emph{ID number} of the reaction, the second column is the \emph{ID
  number} of the species (from \texttt{mechanism.species}) which are
  reactants/products in that reaction:
  \texttt{29\ 71\ 139\ numberOfSpecies\ numberOfReactions\ numberOfGenericComplex\ \ \ 1\ 1\ \ \ 2\ 1\ \ \ 3\ 1\ \ \ 3\ 2}
\item
  \textbf{mechanism.ro2} contains the organic peroxy radicals (RO2). By
  default, it is saved in \texttt{model/configuration/}. The file has a
  comment line header (Fortran style). The first column is the \emph{ID
  number} of the peroxy radical (from \texttt{mechanism.species}), the
  second column is the name of the peroxy radical as Fortran comment:
  \texttt{!\ Note\ that\ this\ file\ is\ generated\ by\ tools/mech\_converter.py\ based\ upon\ the\ file\ tools/mcm\_example.fac.\ Any\ manual\ edits\ to\ this\ file\ will\ be\ overwritten\ when\ calling\ tools/mech\_converter.py\ \ \ 23\ !CH3O2\ \ \ 26\ !C2H5O2\ \ \ 28\ !IC3H7O2\ \ \ 29\ !NC3H7O2}
\end{itemize}

The locations of the files generated during the build process can be
modified by changing the second and the third argument of the script
\texttt{tools/build.sh}. For more information and detailed instructions
go to: {[}{[}2. Model Setup and Execution{]}{]}.

\begin{center}\rule{0.5\linewidth}{\linethickness}\end{center}

\hypertarget{example-mechanism-file}{%
\subsection{Example mechanism file}\label{example-mechanism-file}}

\begin{verbatim}
* ------------------------------------------------------------------- *;
* SIMPLE CHEMICAL MECHANISM                                           *;
* Chemical mechanism for ethanol - from MCM v3.3.1                    *;
* ------------------------------------------------------------------- *;
*;
* Generic Rate Coefficients ;
*;
* Complex reactions ;
*;
* Peroxy radicals. ;
RO2 = HOCH2CH2O2 ;
*;
* Reaction definitions. ;
% 3.0D-12*EXP(20/TEMP)*0.05 : C2H5OH + OH = C2H5O ;
% 3.0D-12*EXP(20/TEMP)*0.9  : C2H5OH + OH = CH3CHO + HO2 ;
% 3.0D-12*EXP(20/TEMP)*0.05 : C2H5OH + OH = HOCH2CH2O2 ;
\end{verbatim}
