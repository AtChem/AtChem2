\section{Environment Variables} \label{sec:environment}

The \textbf{environment variables} define the physical parameters of the
box-model, such as temperature, pressure, humidity, latitude, longitude,
position of the sun, etc\ldots{} These variables are set in the text
file \texttt{model/configuration/environmentVariables.config}.

The environment variables can have a fixed (constant) value or can be
constrained to measured values (\texttt{CONSTRAINED}), in which case the
corresponding data file must be in the
\texttt{model/constraints/environment/} directory (see {[}{[}2.7
Constraints{]}{]}). Some environment variables can be calculated by the
model (\texttt{CALC}) and some can be deactivated if they are not used
by the model (\texttt{NOTUSED}).

By default, most environment variables are set to a fixed value,
corresponding to \emph{standard environmental conditions} (listed
below), or to \texttt{NOTUSED}.

\hypertarget{temp}{%
\subsection{TEMP}\label{temp}}

Ambient Temperature (K).

\begin{itemize}
\item
  fixed value
\item
  constrained
\end{itemize}

Default fixed value = 298.15

\hypertarget{press}{%
\subsection{PRESS}\label{press}}

Ambient Pressure (mbar).

\begin{itemize}
\item
  fixed value
\item
  constrained
\end{itemize}

Default fixed value = 1013.25

\hypertarget{rh}{%
\subsection{RH}\label{rh}}

Relative Humidity (\%). It is required only if \textbf{H2O} is set to
\texttt{CALC}, otherwise should be set to \texttt{NOTUSED}.

\begin{itemize}
\item
  fixed value
\item
  constrained
\item
  not used
\end{itemize}

Default = NOTUSED (-1)

\hypertarget{h2o}{%
\subsection{H2O}\label{h2o}}

Water Concentration (molecules cm-3).

\begin{itemize}
\item
  fixed value
\item
  constrained
\item
  calculated -\textgreater{} requires \textbf{RH} set to fixed value or
  \texttt{CONSTRAINED}
\end{itemize}

Default fixed value = 3.91e+17

\hypertarget{dec}{%
\subsection{DEC}\label{dec}}

Sun Declination (radians) is the angle between the center of the Sun and
Earth's equatorial plane.

\begin{itemize}
\item
  fixed value
\item
  constrained
\item
  calculated -\textgreater{} requires \textbf{DAY} and \textbf{MONTH},
  which are set in \texttt{model.parameters} (see {[}{[}2.2 Model
  Parameters{]}{]})
\end{itemize}

Default fixed value = 0.41

\hypertarget{blheight}{%
\subsection{BLHEIGHT}\label{blheight}}

Boundary Layer Height. It is required only if the model includes
emission or deposition processes (it must be used in the chemical
mechanism as a multiplier of the rate coefficient). The unit is
typically in cm, but it depends on how the processes are parameterized
in the chemical mechanism (see {[}{[}2.1 Chemical Mechanism{]}{]}).

\begin{itemize}
\item
  fixed value
\item
  constrained
\item
  not used
\end{itemize}

Default = NOTUSED (-1)

\hypertarget{dilute}{%
\subsection{DILUTE}\label{dilute}}

Dilution rate. It is required only if the model includes a dilution
process (it must be used in the chemical mechanism as a multiplier of
the rate coefficient). The unit is typically in s-1, but it depends on
how the process is parameterized in the chemical mechanism (see
{[}{[}2.1 Chemical Mechanism{]}{]}).

\begin{itemize}
\item
  fixed value
\item
  constrained
\item
  not used
\end{itemize}

Default value = NOTUSED (-1)

\hypertarget{jfac}{%
\subsection{JFAC}\label{jfac}}

Correction factor used to correct the photolysis rates (e.g., to account
for cloudiness). The calculated photolysis rates are scaled by JFAC,
which can have a value between \texttt{0} (photolysis rates go to zero)
and \texttt{1} (photolysis rates are not corrected). JFAC is NOT applied
to constant or constrained photolysis rates. For more information go to:
{[}{[}2.5 Photolysis Rates and JFAC{]}{]}.

\begin{itemize}
\item
  fixed value
\item
  constrained
\item
  calculated
\end{itemize}

Default fixed value = 1

\hypertarget{roof}{%
\subsection{ROOF}\label{roof}}

Flag to turn the photolysis rates ON/OFF. It is used in simulations of
environmental chamber experiments, where the roof of the chamber can be
opened/closed or the lights turned on/off.

When ROOF is set to \texttt{CLOSED} all the photolysis rates are zero,
including those that are constant or constrained; this is different than
setting JFAC to \texttt{0}, which only applies to the calculated
photolysis rates (see above). ROOF is the only environment variable that
cannot be set to \texttt{CONSTRAINED}.

Default value = OPEN

\begin{center}\rule{0.5\linewidth}{\linethickness}\end{center}

\hypertarget{standard-environmental-conditions}{%
\subsection{Standard environmental
conditions}\label{standard-environmental-conditions}}

\begin{verbatim}
Temperature = 25C  
Pressure = 1 atm  
Relative Humidity = 50%  
Day, Month = 21 June
\end{verbatim}
