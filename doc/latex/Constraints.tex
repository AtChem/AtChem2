\section{Constraints} \label{sec:constraints}

AtChem2 can be run in two modes:

\begin{itemize}
\item
  unconstrained: all variables are calculated by the model from the
  initial conditions, set in the {[}{[}model configuration
  files\textbar{}2.6 Config Files{]}{]}.
\item
  constrained: one or more variables are constrained, i.e.~the solver
  forces their value to a given value. The variables that are not
  constrained are calculated by the model.
\end{itemize}

The constrained values must be provided as separate files for each
constrained variable. The format of the constraint files is described
below. By default, the files with the constraining data are in
\texttt{model/constraints/species/} for the chemical species,
\texttt{model/constraints/environment/} for the environment variables,
and \texttt{model/constraints/photolysis/} for the photolysis rates. The
default directories can be modified by changing the arguments of the
\texttt{atchem2} executable (see {[}{[}2. Model Setup and
Execution{]}{]}).

\hypertarget{constrained-variables}{%
\subsection{Constrained variables}\label{constrained-variables}}

\hypertarget{environment-variables}{%
\subsubsection{Environment variables}\label{environment-variables}}

All environment variables, except \texttt{ROOF}, can be constrained. To
do so, set the variable to \texttt{CONSTRAINED} in
\texttt{model/configuration/environmentVariables.config} and create the
file with the constraining data. The name of the file must be the same
as the name of the variable, e.g.~\texttt{TEMP} (without extension). See
also: {[}{[}2.4 Environment Variables{]}{]}.

\hypertarget{chemical-species}{%
\subsubsection{Chemical species}\label{chemical-species}}

Any chemical species in the chemical mechanism can be constrained. To do
so, add the name of the species to
\texttt{model/configuration/speciesConstrained.config} and create the
file with the constraining data. The name of the file must be the same
as the name of the chemical species, e.g.~\texttt{CH3OH} (without
extension). See also: {[}{[}2.6 Config Files{]}{]}.

\hypertarget{photolysis-rates}{%
\subsubsection{Photolysis rates}\label{photolysis-rates}}

Any of the photolysis rates in the chemical mechanism can be
constrained. The photolysis rates are identified as
\texttt{J\textless{}n\textgreater{}}, where \texttt{n} is an integer
(see {[}{[}2.5 Photolysis Rates and JFAC{]}{]}). To constrain a
photolysis rate add its name (\texttt{Jn}) to
\texttt{model/configuration/photolysisConstrained.config} and create the
file with the constraining data. The name of the file must be the same
as the name of the photolysis rate, e.g.~\texttt{J4} (without
extension). See also {[}{[}2.6 Config Files{]}{]}.

\hypertarget{constraint-files}{%
\subsection{Constraint files}\label{constraint-files}}

The files with the constraining data are text files with two columns
separated by spaces. The first column is the time in \textbf{seconds}
from midnight of day/month/year (see {[}{[}2.2 Model Parameters{]}{]}),
the second column is the value of the variable in the appropriate unit.
For the chemical species the unit is \textbf{molecules cm-3} and for the
photolysis rates the unit is \textbf{s-1}; for the environment variables
see the related {[}{[}wiki page\textbar{}2.4 Environment
Variables{]}{]}. For example:

\begin{verbatim}
-900   73.21
0      74.393
900    72.973
1800   72.63
2700   72.73
3600   69.326
4500   65.822
5400   63.83
6300   64.852
7200   64.739
\end{verbatim}

The time in the first column of a constraint file can be negative.
AtChem2 interprets the negative timestamps as ``seconds \emph{before}
midnight of day/month/year'' (see {[}{[}2.2 Model Parameters{]}{]}).
This can be useful to allow correct interpolation of the variables at
the beginning of the model run (see below).

\textbf{Important.} The constraints must cover the same amount of time,
or preferably more, as the intended model runtime. For example: if the
model starts at 42300 seconds and stops at 216000 seconds, the first and
the last data points in a constraint file must have a timestamp of 42300
(or lower) and 21600 (or higher), respectively.

\hypertarget{interpolation}{%
\subsection{Interpolation}\label{interpolation}}

Constraints can be provided at different timescales. Typically, the
constraining data come from direct measurements and it is a very common
for different instruments to sample at different frequencies. For
example, ozone and nitrogen oxides can be measured once every minute,
but most organic compounds can be measured only once every hour.

The user can average the constraints so that they are all at the same
timescale or can use the data with the original timestamps. Both
approaches have advantages and disadvantages in terms of how much
pre-processing work is required, and in terms of model accuracy and
integration speed. Whether all the constraints have the same timescale
or not, the solver interpolates between data points using the
interpolation method selected in the
\texttt{model/configuration/model.parameters} file (see {[}{[}2.2 Model
Parameters{]}{]}). The default interpolation method is piecewise linear,
but piecewise constant interpolation is also available.

The photolysis rates and the environment variables are evaluated by the
solver when needed - each is interpolated individually, only when
constrained. This happens each time the function
\texttt{mechanism\_rates()} is called from \texttt{FCVFUN()}, and
therefore is controlled by \textbf{CVODE} as it completes the
integration. In a similar way, the interpolation routine for the
chemical species is called once for each of the constrained species in
\texttt{FCVFUN()}, plus once when setting the initial conditions of each
of the constrained species.

As mentioned above, the model start and stop time \emph{must be} within
the time interval of the constrained data to avoid interpolation errors
or model crash. If data is not supplied for the full runtime interval,
then the \emph{final} value will be used for all times both \emph{before
the first data point} and \emph{after the last data point}. This
behaviour is likely to change in future versions, at least to avoid the
situation where the last value is used for all times before the first
(see issue \href{https://github.com/AtChem/AtChem2/issues/294}{\#294}).

A warning is printed for all evaluations outside of the supplied time
interval. Users may find it useful to supply data that covers a short
time \emph{beyond} the final model time, which may be used by the
solver.
