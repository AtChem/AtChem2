% -----------------------------------------------------------------------------
%
% Copyright (c) 2017 Sam Cox, Roberto Sommariva
%
% This file is part of the AtChem2 software package.
%
% This file is covered by the MIT license which can be found in the file
% LICENSE.md at the top level of the AtChem2 distribution.
%
% -----------------------------------------------------------------------------

\chapter{Model Development} \label{ch:development}

% -------------------------------------------------------------------- %
\section{General Information} \label{sec:general-information}

AtChem2 is hosted on the \href{https://github.com}{GitHub} developer
platform. Two versions of the model are available in the
\href{https://github.com/AtChem/AtChem2}{repository} at any given
time:

\begin{description}
\item[Stable version]: is indicated by a version number
  (e.g., \texttt{v1.0}) and can be downloaded from the
  \href{https://github.com/AtChem/AtChem2/releases}{Releases page}.
  A \textbf{doi number} is assigned to each stable version via the
  \href{https://zenodo.org}{Zenodo service}.
\item[Development version]: is the \texttt{master\ branch} of the git
  repository and is indicated by a version number with the suffix
  ``-dev'' (e.g., \texttt{v1.1-dev}). It can be downloaded as an
  \href{https://github.com/AtChem/AtChem2/archive/master.zip}{archive file}
  or obtained via \textbf{git} -- go to Sect.~\ref{sec:download} for
  more information.
\end{description}

AtChem2 is under active development, which means that the
\texttt{master\ branch} is sometimes a few commits ahead of the latest
stable release. However, all code modifications are automatically run
through the \hyperref[sec:test-suite]{Testsuite} before being merged
into the \texttt{master\ branch}. Since the Testsuite is designed to
ensure that changes to the code do not cause unintended behaviour or
unexplained differences in the model results, this ensures that even
the development version of AtChem2 is reliable for most use
cases. Nevertheless, it is strongly suggested to use the
\textbf{stable version} for ``production runs'' and publications, as
it can be easily referenced via the associated doi number -- see the
discussion about traceability and reproducibility of computational
models in \citet{sommariva_2020}.

Feedback, bug reports, comments, suggestions, and feature requests are
welcome: the list of open issues, known bugs, etc\ldots can be found on
the related \href{https://github.com/AtChem/AtChem2/issues}{GitHub page}.
Please check this page before reporting new issues/bugs. Also note
that the AtChem2 \href{https://github.com/AtChem/AtChem2/wiki}{wiki}
contains suggestions to solve some common issues.

The preferred way to contribute to the development of AtChem2 is to
use the open source \href{https://git-scm.com}{git} version control
tool: instructions on how to set up git, prepare and submit
contributions to the AtChem2 repository can be found on the
\href{https://github.com/AtChem/AtChem2/wiki/How-to-contribute}{wiki}.
Contributors are strongly encouraged to follow the Fortran coding
guidelines, as detailed in Sect.\ref{sec:style-guide}.

The AtChem2 user manual is in \LaTeX\ format, with images in svg
format. Contributors should edit

% -------------------------------------------------------------------- %
\section{Testsuite} \label{sec:test-suite}

AtChem2 uses \href{https://github.com/features/actions}{GitHub
  Actions} for \underline{continuous integration} testing. This
programming approach ensures that changes to the code do not
unintentionally modify the behaviour and the results of a
software. The Testsuite is located in the \texttt{tests/}
directory~\footnote{Called \texttt{travis/} in versions 1.0-1.2.2,
  when the TravisCI service was used to run the Testsuite.}  and
consists of a series of tests and short model runs that check the
model functionality and calculations against known outputs. In
addition, the \href{https://codecov.io}{Codecov} service is used to
ensure that the Testsuite covers a significant fraction of the
codebase ($>$90\%) and a wide range of common configurations.

There are four types of tests, executed from the \maindir\ with
the \verb|make| command. Note that the \emph{optional dependencies}
must be installed in order to run the Testsuite -- see
Sect.~\ref{subsec:optional-dependencies} and \ref{subsec:tests-optional}
for details.

\begin{itemize}
\item the \textbf{Indent} and \textbf{Style} tests check that the
  indentation and coding style of the Fortran code are consistent with
  the guidelines:\\
  $\hookrightarrow$ \verb|make indenttest| and \verb|make styletest|
\item the \textbf{Unit} tests check that individual Fortran functions
  generate the expected outputs:\\
  $\hookrightarrow$ \verb|make unittests|
\item the \textbf{Model} tests run a number of models with different
  configurations to check that they generate the expected outputs:\\
  $\hookrightarrow$ \verb|make modeltests| and/or
  \verb|make oldtests|~\footnote{At present these tests are split into
    two groups, \texttt{modeltests} and \texttt{oldtests}, which will
    eventually be merged. Both groups of tests pass on Linux, but
    \texttt{modeltests} fail on macOS. See issue
    \href{https://github.com/AtChem/AtChem2/issues/522}{\#522}.}
\item The command \verb|make alltests| runs all the tests described
  above in succession.
\end{itemize}

The Indent, Style and Model tests create \texttt{.log} files in the
\texttt{tests/} directory and print a summary message to the terminal,
while the Unit tests output the results directly to the terminal. If
one or more tests fail, search the log files for detailed information.

The Testsuite is automatically run every time a Pull Request (PR) is
created or updated on the GitHub repository. The PR triggers a GitHub
Actions workflow which runs the entire Testsuite on two architectures
(Linux and macOS) using different versions of the GNU
\texttt{gfortran} compiler. The workflow -- described in the
\texttt{.github/workflows/ci.yml} file -- performs the following tasks
on each architecture:

\begin{itemize}
\item Install \texttt{gfortran} and the AtChem2 dependencies using
  \texttt{apt-get} (Linux) or \texttt{Homebrew} (macOS).
\item Build and run a model using the example chemical mechanism and
  default configuration: PASS if it exits with 0.
\item Build and run the Indent and Style tests: PASS if all tests pass.
\item Build and run the Unit tests: PASS if all Unit tests pass.
\item Build and run the Model tests: PASS if there are no differences
  from the reference output files (within tolerance), otherwise FAIL.
\item Build and run the Unit and Model tests, and upload the overage
  reports to the Codecov service.
\end{itemize}

The Testsuite is successful only if all tests pass successfully. This is
indicated by the message ``All checks have passed'' on the GitHub PR page.
Pull Requests should only be merged into the \texttt{master\ branch}
once GitHub Actions has completed with passes on both architectures.

\subsection{Adding new Unit tests} \label{subsec:adding-new-unit-tests}

The Unit tests are in the \texttt{tests/unit\_tests/} directory and
require the FRUIT optional dependency (which needs Ruby v3.0,
Sect.~\ref{subsec:optional-dependencies}). To add new Unit tests,
follow the procedure outlined below:

\begin{itemize}
\item The Unit test files are called \texttt{*\_test.f90}. If the new
  Unit test to be added fits into an existing test file, edit that
  file -- otherwise, create a new test file following the same naming
  pattern. It is suggested that Unit tests covering functions from the
  source file \texttt{xFunctions.f90} should be named
  \texttt{x\_test.f90}.
\item The Unit test file must contain a module with the same name as
  the file -- i.e. \texttt{x\_test} -- and it must include the
  statement \verb|use fruit|, plus any other required module.
\item The module should contain a number of subroutines with the
  naming pattern \texttt{test\_*}. These subroutines must take no
  arguments and, importantly, must not have any brackets for arguments
  -- subroutine \texttt{test\_calc} is correct, but subroutine
  \texttt{test\_calc()} is wrong.
\item Each subroutine should call one or more assert functions:
  usually, these are \texttt{assert\_equals()}, \texttt{assert\_not\_equals()},
  \texttt{assert\_true()}, \texttt{assert\_false()}. The assert functions
  act as the arbiters of pass or failure of the Unit test -- each
  assert must pass for the subroutine to pass, and each subroutine
  must pass for the Unit tests to pass.
\item The assert functions have the following syntax:
  \begin{verbatim}
  call assert_true( a == b , "Test that a and b are equal")
  call assert_false( a == b , "Test that a and b are not equal")
  call assert_equals( a, b , "Test that a and b are equal")
  call assert_not_equals( a, b , "Test that a and b are not equal")
  \end{verbatim}
\end{itemize}

It is useful to use the last argument of the assert function as a
\emph{unique} and \emph{descriptive} message. When a Unit test fails,
it is highlighted in the FRUIT output summary, and the message of the
assert function is printed. Unique and descriptive messages thus
enable faster and easier understanding of which test has failed,
and perhaps why.

If these steps are followed, calling \texttt{make\ unittests} is
enough to run all the Unit tests, including the new ones. To verify
that the new tests have indeed been run and passed, check the output
summary -- there should be a line associated to each of the
\texttt{test\_*} subroutines in each test file.

\subsection{Adding new Model tests} \label{subsec:adding-new-model-tests}

Each Model test (\texttt{\$TESTNAME}) is contained in its own
subdirectory inside the \texttt{tests/model\_tests/} directory. A
Model test requires the following files and directory structure:

\begin{verbatim}
|- $TESTNAME
|  |- configuration
|  |  |- customRateFuncs.f90           [*]
|  |  |- environmentVariables.config
|  |  |- initialConcentrations.config
|  |  |- mechanism.prod.cmp
|  |  |- mechanism.reac.cmp
|  |  |- mechanism.ro2.cmp
|  |  |- mechanism.species.cmp
|  |  |- model.parameters
|  |  |- outputRates.config
|  |  |- outputSpecies.config
|  |  |- photolysisConstant.config     [*]
|  |  |- photolysisConstrained.config  [*]
|  |  |- solver.parameters
|  |  |- speciesConstant.config        [*]
|  |  |- speciesConstrained.config     [*]
|  |- constraints      [**]
|     |- environment/  [**]
|     |- photolysis/   [**]
|     |- species/      [**]
|- output
|  |- reactionRates/
|  |- environmentVariables.output.cmp
|  |- errors.output.cmp
|  |- finalModelState.output.cmp
|  |- jacobian.output.cmp
|  |- lossRates.output.cmp
|  |- mainSolverParameters.output.cmp
|  |- photolysisRates.output.cmp
|  |- photolysisRatesParameters.output.cmp
|  |- productionRates.output.cmp
|  |- speciesConcentrations.output.cmp
|- $TESTNAME.{fac|kpp}
|- $TESTNAME.out.cmp
\end{verbatim}

The files marked with \texttt{[*]} and the directories marked with
\texttt{[**]} are optional, depending on the model configuration used
in the test. If present, the directories marked with \texttt{[**]}
should contain the relevant constraint files, according to the
corresponding configuration files in \texttt{model/configuration/}
(see Sect.~\ref{sec:constraints} for details). The file
\texttt{\$TESTNAME.out.cmp} should contain the exact copy of the
expected terminal printout.

Each Model test is briefly described in the \texttt{tests/model\_tests/INFO.md}
file, which should be updated after a new test is added. New tests added
to the \texttt{tests/tests/} directory are automatically picked up when
running the commands \verb|make modeltests| or \verb|make alltests| from
the \maindir.

% -------------------------------------------------------------------- %
\section{Style Guide} \label{sec:style-guide}

In order to make the AtChem2 code more readable and easier to
maintain, the Fortran source code should follow a consistent style
(Sect.~\ref{subsec:style-recommendations}). Two Python scripts are
available to check and, if necessary, correct the Fortran code:

\begin{itemize}
\item \texttt{fix\_style.py} edits a Fortran file in-place to make the
  code consistent with the style recommendations.
\item \texttt{fix\_indent.py} works in a similar way, but only looks
  at the indentation level of each line of code.
\end{itemize}

These scripts are in the \texttt{tools/} directory and are executed
from the \maindir\ with the following commands:

\begin{verbatim}
python tools/fix_style.py src/filename.f90
python tools/fix_indent.py src/filename.f90
\end{verbatim}

It is important to keep in mind that these scripts are \emph{not infallible}
and, therefore, it is strongly recommended to always have a backup of the
original Fortran file to revert to, in case of wrong edits. This can be
done by passing two arguments to the script instead of one: the second
argument sends the script output to another file, leaving the original
file untouched.

Both scripts are also used in the Testsuite to run the Style and
Indent tests (Sect.~\ref{sec:test-suite}): each script is run over
each source file and the output is sent to a \texttt{.cmp} file. If
all \texttt{.cmp} files match the original files, the test passes.

\subsection{Style recommendations} \label{subsec:style-recommendations}

\subsubsection{General principles}

\begin{itemize}
\item All code should be organized in a \textbf{module structure},
  except the main program. There is only one exception: due to an
  issue with linking to CVODE, the functions \texttt{FCVFUN()} and
  \texttt{FCVJTIMES()} are placed within the main file
  (\texttt{atchem.f90}).
\item All code should be written in free-form Fortran, compliant with
  the Fortran 90/95 standard; the source files should have the
  extension \texttt{.f90}.
\item Always use two spaces to indent blocks.
\item At the top of each file there should be a header indicating the
  author(s), date, and purpose of the code. If applicable,
  acknowledgements to other contributors should be added.
\item Always comment a procedure with a high-level explanation of what
  that procedure does.
\item There are no specific guidelines for comments, although common
  sense applies, and any code within the comments should broadly
  follow the rules below.
\end{itemize}

\subsubsection{Specific recommendations}

\begin{itemize}
\item All \textbf{keywords} should be lowercase: e.g., \texttt{if\ then},
  \texttt{call}, \texttt{module}, \texttt{integer}, \texttt{real},
  \texttt{only}, \texttt{intrinsic}. This includes the
  \texttt{(kind=XX)} and \texttt{(len=XX)} statements.
\item All \textbf{intrinsic} function names should be lowercase: e.g.,
  \texttt{trim}, \texttt{adjustl}, \texttt{adjustr}.
\item The \textbf{relational operators} should use \texttt{$\geq$} and
  \texttt{==} rather than \texttt{.GE.}, \texttt{.EQ.} -- they should
  be surrounded by a single space.
\item The \texttt{=} operator should be surrounded by one space when
  used as assignment, except in the cases of the \texttt{(kind=XX)}
  and \texttt{(len=XX)} statements where no spaces should be used.
\item The \textbf{mathematical operators} (\texttt{*}, \texttt{-},
  \texttt{+}, \texttt{**}) should be surrounded by one space.
\item Numbers in scientific notation should have no spaces around the
  \texttt{+} or \texttt{-} signs: e.g., \texttt{1.5e-9}.
\item The names of the \textbf{variables} should begin with lowercase,
  while those of the \textbf{procedures} (subroutines and functions)
  should begin with uppercase. An exception to this rule is
  \textbf{third-party functions}, which should be uppercase. Both
  CamelCase or underscores are acceptable for multiple-word
  identifiers, but be consistent.
\item All \textbf{modules} should include the \texttt{implicit none}
  statement.
\item All \textbf{variable declarations} should include the
  \texttt{::} notation.
\item The \textbf{dummy arguments} of a procedure should include an
  \texttt{intent} statement in their declaration.
\item The following rules apply to \textbf{brackets}:
  \begin{itemize}
  \item Opening brackets should not have a space before them, except
    for \texttt{read}, \texttt{write}, \texttt{open}, \texttt{close}
    statements.
  \item All \texttt{call} statements and the definitions of all
    procedures should contain one space before the first argument and
    one after the last argument inside the brackets:
    \begin{verbatim}
    call function_name( arg1, arg2 )
    subroutine subroutine_name( arg1 )
    \end{verbatim}
  \item Functions calls and array indices should not have spaces
    before the first argument or after the last argument inside the
    brackets.
  \end{itemize}
\end{itemize}
