\section{Installation} \label{sec:install}

AtChem2 can be installed on Linux/Unix or macOS. A working knowledge of
the \textbf{unix shell} and its
\href{https://swcarpentry.github.io/shell-novice/reference/}{basic
commands} is \emph{required} to install and use the model.

\hypertarget{download}{%
\subsection{Download}\label{download}}

There are two versions of AtChem2: the
\href{https://github.com/AtChem/AtChem2/releases}{stable version} and
the development version, also known as \texttt{master\ branch} (see the
{[}{[}model development page\textbar{}3. Model Development{]}{]} for
additional information). The source code can be obtained in two ways:

\begin{enumerate}
\def\labelenumi{\arabic{enumi}.}
\item
  with \textbf{git}:

  \begin{enumerate}
  \def\labelenumii{\arabic{enumii}.}
    \item
    Open the terminal. Move to the directory where you want to install
    AtChem2.
  \item
    Execute \texttt{git\ clone\ https://github.com/AtChem/AtChem2.git}
    (if using HTTPS) or
    \texttt{git\ clone\ git@github.com:AtChem/AtChem2.git} (if using
    SSH). This method will download the development version and it is
    recommended if you want to contribute to the model development.
  \end{enumerate}
\item
  with the \textbf{archive file} (\texttt{*.tar.gz} or \texttt{*.zip}):

  \begin{enumerate}
  \def\labelenumii{\arabic{enumii}.}
    \item
    Download the archive file of the stable version
    (https://github.com/AtChem/AtChem2/releases) or of the development
    version (https://github.com/AtChem/AtChem2/archive/master.zip) to
    the directory where you want to install AtChem2.
  \item
    Open the terminal. Move to the directory where you downloaded the
    archive file.
  \item
    Unpack the archive file (e.g., \texttt{tar\ -zxvf\ v1.1.tar.gz} or
    \texttt{unzip\ master.zip}).
  \end{enumerate}
\end{enumerate}

Depending on which of these methods you have used, the source code is
now in a directory called \texttt{AtChem2} or \texttt{AtChem2-1.1} or
\texttt{AtChem2-master}. This directory - which you can rename, if you
want to - is the \emph{main directory} of the model. In the
documentation we will assume that the \emph{AtChem2 main directory} is
\texttt{\$HOME/AtChem2}.

\hypertarget{requirements}{%
\subsection{Requirements}\label{requirements}}

AtChem2 needs the following tools:

\begin{itemize}
\item
  a Fortran compiler: the model compiles with GNU \texttt{gfortran}
  (version 4.8.5) and with Intel \texttt{ifort} (version 17.0)
\item
  Python 2.7.x
\item
  cmake
\item
  Ruby 2.0 (optional)
\end{itemize}

Some or all of these tools may already be present on your system. Use
the \texttt{which} command to find out (e.g., \texttt{which\ python},
\texttt{which\ cmake}, etc\ldots{}). Otherwise, check the local
documentation or ask the system administrator.

In addition, AtChem2 has the following dependencies:

\begin{itemize}
\item
  the CVODE library
\item
  the openlibm library
\item
  the BLAS and LAPACK libraries
\item
  numdiff (optional)
\item
  FRUIT (optional)
\end{itemize}

For detailed instructions on the installation and configuration of the
dependencies go to: {[}{[}1.1 Dependencies{]}{]}.

\hypertarget{install}{%
\subsection{Install}\label{install}}

To install AtChem2:

\begin{enumerate}
\def\labelenumi{\arabic{enumi}.}
\item
  Move to the \emph{AtChem2 main directory}
  (\texttt{cd\ \textasciitilde{}/AtChem2/}). Install the
  {[}{[}dependencies\textbar{}1.1 Dependencies{]}{]} and take note of
  the name and path of the \emph{dependencies directory} (in the
  following instructions, we will assume that the \emph{dependencies
  directory} is \texttt{\textasciitilde{}/atchem-libraries/}).
\item
  Copy the \texttt{Makefile} in the \texttt{tools/} directory to the
  \emph{main directory} (\texttt{cp\ tools/Makefile\ ./}).
\item
  From the the \emph{main directory}, open the \texttt{Makefile} with a
  text editor. Set the variables \texttt{CVODELIB},
  \texttt{OPENLIBMDIR}, \texttt{FRUITDIR} to the paths of the CVODE,
  openlibm and FRUIT libraries, as described in the {[}{[}dependencies
  page\textbar{}1.1 Dependencies{]}{]}. Use the full path to the
  libraries, not the relative path (see issue
  \href{https://github.com/AtChem/AtChem2/issues/364}{\#364}). For
  example:
  \texttt{CVODELIB\ \ \ \ \ =\ \$(HOME)/atchem-libraries/cvode/lib\ \ OPENLIBMDIR\ \ =\ \$(HOME)/atchem-libraries/openlibm-0.4.1\ \ FRUITDIR\ \ \ \ \ =\ \$(HOME)/atchem-libraries/fruit\_3.4.3}
\item
  Execute \texttt{./tools/build.sh\ ./tools/mcm\_example.fac}. This
  command compiles the model and creates an executable
  (\texttt{atchem2}) using the test mechanism file
  \texttt{mcm\_example.fac} in the \texttt{tools/} directory.
\item
  Execute \texttt{./atchem2}. If the model has been installed correctly,
  you should see a message similar to this: ``` ------------------ Final
  statistics ------------------ No.~steps = 546 No.~f-s = 584 No.~J-s =
  912 No.~LU-s = 56 No.~nonlinear iterations = 581 No.~nonlinear
  convergence failures = 0 No.~error test failures = 4

  Runtime = 0 Deallocating memory. ```
\end{enumerate}

This means that AtChem2 has completed the test run without errors and is
ready to be used. The directory structure of AtChem2 is described
{[}{[}here\textbar{}1.2 Model Structure{]}{]}. For instructions on how
to set up, compile and execute the model go to: {[}{[}2. Model Setup and
Execution{]}{]}.

\hypertarget{note-for-macos-users}{%
\subsubsection{Note for macOS users}\label{note-for-macos-users}}

When you first run AtChem2, you may receive an error message like this:

\begin{verbatim}
dyld: Library not loaded: @rpath/libsundials_cvode.2.dylib
Referenced from: /Users/username/AtChem2/./atchem2
Reason: image not found
Abort trap: 6
\end{verbatim}

In this case, type at the terminal prompt the following command (change
the path to the CVODE library as appropriate):

\begin{verbatim}
export DYLD_LIBRARY_PATH=$(HOME)/atchem-libraries/cvode/lib
\end{verbatim}

To make it permanent, add the command to your
\texttt{\textasciitilde{}/.bash\_profile} file. Advanced users may wish
to use instead the accepted answer in
\href{https://stackoverflow.com/questions/17703510/dyld-library-not-loaded-reason-image-not-loaded}{this
Stack Overflow post} to hardcode \texttt{rpath} in this instance for
each of \texttt{libsundials\_cvode.2.dylib},
\texttt{libsundials\_fvecserial.2.dylib},
\texttt{libsundials\_vecserial.2.dylib}.

\hypertarget{tests-optional}{%
\subsection{Tests (optional)}\label{tests-optional}}

You can run the {[}{[}Test Suite\textbar{}3.1 Test Suite{]}{]} to verify
that AtChem2 has been installed properly and to make sure that changes
to the code do not result in unintended behaviour. This is recommended
if you want to contribute to the model development. Note that running
the Test Suite requires the optional dependencies to be installed, as
explained in the {[}{[}dependencies page\textbar{}1.1
Dependencies{]}{]}.

To run the tests, execute the following commands from the \emph{AtChem2
main directory}: * \texttt{make\ alltests} runs all the tests (requires
\textbf{numdiff} and \textbf{FRUIT}) * \texttt{make\ tests} runs only
the behaviour tests (requires \textbf{numdiff}) *
\texttt{make\ unittests} runs only the unit tests (requires
\textbf{FRUIT})

For more information on the Test Suite go to the corresponding
{[}{[}wiki page\textbar{}3.1 Test Suite{]}{]}.
