% -----------------------------------------------------------------------------
%
% Copyright (c) 2017 Sam Cox, Roberto Sommariva
%
% This file is part of the AtChem2 software package.
%
% This file is covered by the MIT license which can be found in the
% file LICENSE.md at the top level of the AtChem2 distribution.
%
% -----------------------------------------------------------------------------
\chapter{Model Execution} \label{ch:execution}

% -------------------------------------------------------------------- %
\section{The Box-model} \label{sec:boxmodel}

AtChem2 is designed to build and run atmospheric chemistry box-models
based upon the Master Chemical Mechanism
(\href{http://mcm.leeds.ac.uk/MCM/}{MCM}). This page explains how to
set up, compile and run an atmospheric chemistry box-model with
AtChem2. The directory structure of AtChem2 is described in
\hyperref[sec:structure]{Model Structure}. A working knowledge of the
\textbf{unix shell} and its
\href{https://swcarpentry.github.io/shell-novice/reference/}{basic
  commands} is \emph{required} to use the AtChem2 model.

There are two sets of inputs to AtChem2 -- the mechanism file, and
configuration files.

\subsection{Mechanism file} \label{subsec:mechanism-file}

The model requires a chemical mechanism in FACSIMILE format
(\texttt{.fac}). The \textbf{mechanism file} can be downloaded from
the MCM website using the
\href{http://mcm.leeds.ac.uk/MCMv3.3.1/extract.htt}{extraction tool}
or assembled manually. The user can modify the \texttt{.fac} file as
required with a text editor. This mechanism file is converted into a
shared library and a set of associated data files in the compilation
step below. For more information on the chemical mechanism go to:
\hyperref[sec:mechanism]{Chemical Mechanism}.

\subsection{Configuration files} \label{subsec:configuration-files}

The \textbf{model configuration} is set via a number of text files
located in the \texttt{model/configuration/} directory. The text files
can be modified with a text editor. Detailed information on the
configuration files can be found in the corresponding wiki pages:

\begin{itemize}
\item model and solver parameters -- see
  \hyperref[sec:parameters]{Model Parameters} and
  \hyperref[sec:solver]{Solver Parameters}.
\item environment variables -- see \hyperref[sec:envvar]{Environment
    Variables}.
\item photolysis rates -- see \hyperref[sec:photolysis]{Photolysis
    Rates and JFAC}.
\item initial concentrations of chemical species and lists of output
  variables -- see \hyperref[sec:config]{Config Files}.
\end{itemize}

The model constraints -- chemical species, environment variables,
photolysis rates -- are located in the \texttt{model/constraints/}
directory. For more information, go to:
\hyperref[sec:constraints]{Constraints}.

% -------------------------------------------------------------------- %
\section{Building the model} \label{sec:build}

\subsection{Compilation} \label{subsec:compilation}

The script \texttt{build\_atchem2.sh} in the \texttt{tools/} directory is used
to process the chemical mechanism file (\texttt{.fac}) and to compile
the model. The script generates one Fortran file
(\texttt{mechanism.f90}), one shared library (\texttt{mechanism.so})
and four configuration files (\texttt{mechanism.prod},
\texttt{mechanism.reac}, \texttt{mechanism.ro2},
\texttt{mechanism.species}) in the \texttt{model/configuration/}
directory. Go to the \hyperref[sec:mechanism]{Chemical Mechanism} for
more information.

The script must be run from the \emph{AtChem2 main directory} and
takes four arguments (see \hyperref[important-note-1]{Important Note
  1}):

\begin{enumerate}
\item the path to the chemical mechanism file -- no default
  (suggested: \texttt{model/}).
\item the path to the directory for the Fortran files generated from
  the chemical mechanism -- default:\\ \texttt{model/configuration/}.
\item the path to the directory with the configuration files -
  default:\\ \texttt{model/configuration/}.
\item the path to the directory with the MCM data files -- default:\\
  \texttt{mcm/}.
\end{enumerate}

For example, if the \texttt{.fac} file is in the \texttt{model/}
directory:

\begin{verbatim}
./build/build_atchem2.sh model/mechanism.fac model/configuration/
                 model/configuration/ mcm/
\end{verbatim}

An installation of AtChem2 can have multiple \texttt{model/}
directories, which may correspond to different models or different
projects; this allows the user to run more than one model at the same
time. In the following example, there are two \texttt{model/}
directories, each with their own chemical mechanism, configuration,
constraints and output:

\newpage
\begin{verbatim}
AtChem2/
        | mcm/
        | model_1/
             | configuration/
             | constraints/
             | output/
             | mechanism.fac
        | model_2/
             | configuration/
             | constraints/
             | output/
             | mechanism.fac
        | obj/
        | src/
        | tools/
        | travis/
\end{verbatim}

Each model can be built by passing the correct path to the
\texttt{build\_atchem2.sh} script (see \hyperref[important-note-1]{Important
  Note 1})). For example:

\begin{verbatim}
./build/build_atchem2.sh model_1/mechanism.fac model_1/configuration/
                 model_1/configuration/ mcm/
./build/build_atchem2.sh model_2/mechanism.fac model_2/configuration/
                 model_2/configuration/ mcm/
\end{verbatim}

Compilation is required only once for a given \texttt{.fac} file. If
the user changes the configuration files, there is no need to
recompile the model. Likewise, if the constraints files are changed,
there is no need to recompile the model. This is because the model
configuration and the model constraints are read by the executable at
runtime. However, if the user makes changes to the \texttt{.fac} file,
then the shared library \texttt{model/configuration/mechanism.so}
needs to be recompiled from the source file
\texttt{model/configuration/mechanism.f90} using the \texttt{build\_atchem2.sh}
script.

The user may want or need to change the Fortran code
(\texttt{src/*.f90}), in which case the model needs to be recompiled:
if the \texttt{.fac} file has also been changed, use the
\texttt{build\_atchem2.sh} script, as explained above. Otherwise, if only the
Fortran code has been changed, executing \texttt{make} from the
\emph{main directory} is enough to recompile the model.

\subsubsection{Important Note 1} \label{important-note-1}

The arguments need to be passed to the \texttt{build/build\_atchem2.sh}
executable in the exact order, as listed above. This means that if --
for example -- the third argument needs to be specified, it is also
necessary to specify the first and the second arguments, even if they
have the default values. To avoid mistakes, the user can choose to
always specify all the arguments. Future versions of AtChem2 will
adopt a simpler command-line interface.

\subsection{Execution} \label{subsec:execution}

The compilation process creates an executable file called
\texttt{atchem2} in the \emph{main directory}. The executable file
takes seven arguments, corresponding to the directories containing the
model configuration and output:

\begin{enumerate}
\item the path to the directory for the model output -- default:\\
  \texttt{model/output}
\item the path to the directory for the model output reaction rates -
  default:\\ \texttt{model/output/reactionRates/}
\item the path to the directory with the configuration files -
  default:\\ \texttt{model/configuration/}.
\item the path to the directory with the MCM data files -- default:\\
  \texttt{mcm/}.
\item the path to the directory with the data files of constrained
  chemical species -- default:\\ \texttt{model/constraints/species/}
\item the path to the directory with the data files of constrained
  environment variables -- default:\\
  \texttt{model/constraints/environment/}
\item the path to the directory with the data files of constrained
  photolysis rates -- default:\\
  \texttt{model/constraints/photolysis/}
\end{enumerate}

The model can be run by executing the \texttt{atchem2} command from
the \emph{main directory}, in which case the executable will use the
default configuration and output directories. Otherwise, the
configuration and output directories need to be specified. The
\texttt{atchem2} executableuses the following flags to set the
arguments: \texttt{--model}, \texttt{--output},
\texttt{--reactionRates}, \texttt{--configuration},
\texttt{--constraints}, \texttt{--env\_constraints},
\texttt{--photo\_constraints}, \texttt{--spec\_constraints},
\texttt{--mcm}, and \texttt{--shared-lib}. The command \texttt{atchem2
  --help} displays a help message showing the usage of the command
line arguments.

For example, if the constraints are in the default directories (or not
used), the model can be run by executing:

\begin{verbatim}
./atchem2 --output=model/output/
          --reactionRates=model/output/reactionRates/
          --configuration=model/configuration/
          --spec_constraints=model_1/constraints/species/
          --env_constraints=model_1/constraints/environment/
          --photo_constraints=model_1/constraints/photolysis/
          --mcm=mcm/
\end{verbatim}

In the case of multiple \texttt{model/} directories, the directories
corresponding to each model need to be passed as arguments to the
\texttt{atchem2} executable. This allows the user to run two or more
models simultaneously. For example:

\begin{verbatim}
    ./atchem2 --output=model_1/output/
              --configuration=model_1/configuration/
              --constraints=model_1/constraints/
    ./atchem2 --output=model_2/output/
              --configuration=model_2/configuration/
              --constraints=model_2/constraints/
\end{verbatim}

As explained above, if the chemical mechanism (\texttt{.fac}) is
changed, only the shared library needs to be recompiled. This allows
the user to have only one base executable called \texttt{atchem2} in
the \emph{main directory}: when running multiple models at the same
time the user can reuse this base executable while pointing each model
to the correct shared library and configuration files.

\subsection{Output} \label{subsec:output}

The model output is saved by default in the directory
\texttt{model/output/}. The location can be modified by changing the
arguments of the \texttt{atchem2} executable (see above).

The AtChem2 output files are space-delimited text files, with a header
containing the names of the variables:

\begin{itemize}
\item values of environment variables and concentrations of chemical
  species:\\ \texttt{environmentVariables.output},
  \texttt{speciesConcentrations.output}.
\item values of photolysis rates and related parameters:\\
  \texttt{photolysisRates.output},
  \texttt{photolysisRatesParameters.output}.
\item loss and production rates of selected species (see
  \hyperref[sec:config]{Config Files}):\\ \texttt{lossRates.output},
  \texttt{productionRates.output}.
\item Jacobian matrix (if requested, see
  \hyperref[sec:parameters]{Model Parameters}):\\
  \texttt{jacobian.output}.
\item model diagnostic variables:\\ \texttt{finalModelState.output},
  \texttt{initialConditionsSetting.output},
  \texttt{mainSolverParameters.output}.
\end{itemize}

In addition, the reaction rates of all the reactions in the chemical
mechanism are saved in the directory \texttt{reactionRates/}: one file
for each model step, with the filename corresponding to the time in
seconds.
