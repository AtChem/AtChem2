% -----------------------------------------------------------------------------
%
% Copyright (c) 2017 Sam Cox, Roberto Sommariva
%
% This file is part of the AtChem2 software package.
%
% This file is covered by the MIT license which can be found in the file
% LICENSE.md at the top level of the AtChem2 distribution.
%
% -----------------------------------------------------------------------------

\chapter{Model Development} \label{ch:development}

% -------------------------------------------------------------------- %
\section{General Information} \label{sec:general-information}

Two versions of AtChem2 are available:

\begin{enumerate}
\item the stable version, which is indicated by a version number
  (e.g., \textbf{v1.0}), and can be found
  \href{https://github.com/AtChem/AtChem2/releases}{here}.
\item the development version: which is indicated by a version number
  with the suffix \texttt{-dev} (e.g., \textbf{v1.1-dev}), and can be
  downloaded from the
  \href{https://github.com/AtChem/AtChem2/archive/master.zip}{master
    branch} or obtained via \textbf{git}.
\end{enumerate}

AtChem2 is under active development, which means that the
\texttt{master\ branch} may sometimes be a few steps ahead of the
latest stable release. The \hyperref[sec:test-suite]{Test Suite} is
designed to ensure that changes to the code do not cause unintended
behaviour or unexplained differences in the model results, so the
development version is usually safe to use, although caution is
advised.

The roadmap for the development of AtChem2 can be found
\href{https://github.com/AtChem/AtChem2/projects/1}{here}.

Feedback, bug reports, comments and suggestions are welcome. Please
check \href{https://github.com/AtChem/AtChem2/issues}{this github
  page} for a list of known and open issues.

If you want to contribute to the model development, the best way is to
use \textbf{git}. The procedure to contribute code is described
on this \href{https://github.com/AtChem/AtChem2/wiki/How-to-contribute}{wiki page}.
A basic level of \href{https://swcarpentry.github.io/git-novice/}{knowledge of git}
is \emph{required}.

% -------------------------------------------------------------------- %
\section{Test Suite} \label{sec:test-suite}

AtChem2 uses \href{https://travis-ci.org/}{Travis CI} for Continuous
Integration testing. This programming approach ensures changes to the
code do not modify the behaviour and the results of the software in an
unintended fashion.

To begin using CI on code modifications, create a Pull Request on
github from your own fork to \texttt{AtChem/AtChem2} (see
\hyperref[sec:general-information]{previous section} for instructions on how
to set up \textbf{git}). Once the PR is created, Travis CI will
automatically run build, unit and behaviour tests on 2 architectures
(linux and OSX). Pull requests should only be merged once the Travis
CI has completed with passes on both architectures. This is indicated
by the message: ``All checks have passed''.

In order to run the Test Suite on your local machine, call
\texttt{make\ alltest} from the \maindir. This will run
each of the 3 classes of test in this order:

\begin{itemize}
\item unit tests: checks that small fragments of code generate the
  expected outputs;
\item build test: checks that an example program builds and runs
  successfully;
\item behaviour tests: builds each of a number of test setups in turn,
  and checks that they generate the expected outputs.
\end{itemize}

Each of the test classes outputs the results of their tests to the
terminal screen. To perform just the unit tests, call \texttt{make\
  unittests}. To run just the build and behaviour tests, call
\texttt{make\ tests}.

The CI tester performs the following on each architecture:

\begin{itemize}
\item Install \texttt{gfortran}, \texttt{cvode}, and \texttt{numdiff}
  \begin{itemize}
  \item linux: use \texttt{apt-get} for \texttt{gfortran},
    \texttt{numdiff}, and \texttt{liplapack-dev} (a dependency of
    \texttt{cvode}). Install \texttt{cvode} from source
    (\texttt{apt-get} could also be used to install \texttt{sundials}
    (including \texttt{cvode}), but it doesn't currently hold
    \texttt{cvode\ 2.9}).
  \item OSX: use Homebrew for \texttt{gfortran} and
    \texttt{numdiff}. Install \texttt{cvode} from source.
  \end{itemize}
\item Build and run unit tests. PASS if all unit tests pass.
\item Build and run a single example of AtChem2. PASS if this exits
  with 0.
\item Build and run several other examples of AtChem2, using different
  input files. PASS if no differences from the reference output files
  are found, otherwise FAIL. Every test must pass to allow the full CI
  to PASS.
\end{itemize}

\subsection{Adding new unit tests} \label{subsec:adding-new-unit-tests}

To add new unit tests, do the following:

\begin{itemize}
\item Navigate to \texttt{travis/unit\_tests}. This contains several
  files with the ending \texttt{*\_test.f90}. IF the new test to be
  added fits into an existing test file, edit that file -- otherwise,
  make a new file, but it must follow that pattern of
  \texttt{*\_test.f90}. It is suggested that unit tests covering
  functions from the source file \texttt{xFunctions.f90} should be
  named \texttt{x\_test.f90}.
\item The file must contain a module with the same name as the file,
  i.e., \texttt{*\_test}. It must \texttt{use\ fruit}, and any other
  modules as needed.
\item The module should contain subroutines with the naming scheme
  \texttt{test\_*\textasciitilde}. These subroutines must take no
  arguments (and, crucially, not have any brackets for arguments
  either -- subroutine \texttt{test\_calc} is correct, but subroutine
  \texttt{test\_calc()} is wrong).
\item Each subroutine should call one or more assert functions
  (usually \texttt{assert\_equals()}, \texttt{assert\_not\_equals()},
  \texttt{assert\_true()} or \texttt{assert\_false()}). These assert
  functions act as the arbiters of pass or failure -- each assert must
  pass for the subroutine to pass, and each subroutine must pass for
  the unit tests to pass.
\item The assert functions have the following syntax:

\begin{verbatim}
call assert_true( a == b , "Test that a and b are equal")
call assert_false( a == b , "Test that a and b are not equal")
call assert_equals( a, b , "Test that a and b are equal")
call assert_not_equals( a, b , "Test that a and b are not equal")
\end{verbatim}

\end{itemize}

It is useful to use the last argument as a \emph{unique} and
\emph{descriptive} test message. If any unit tests fail, then this
will be highlighted in the summary, and the message will be
printed. Unique and descriptive messages enable faster and easier
understanding of which test has failed, and perhaps why.

If these steps are followed, calling \texttt{make\ unittests} is
enough to run all the unit tests, including new ones. To check that
your new tests have indeed been run and passed, check the output
summary -- you should see a line associated to each of the
\texttt{test*} subroutines in each file in the suite of unit tests.

\subsection{Adding new behaviour tests} \label{subsec:adding-new-behaviour-tests}

To add a new behaviour test called \texttt{\$TESTNAME} to the
Test Suite, you should provide the following:

Each input \texttt{\$TESTNAME} should have a subdirectory
\texttt{travis/tests/\$TESTNAME/} containing the following files in
the following structure (\texttt{*} indicates that this file/directory
is optional dependent on the configuration used in the test, while
\texttt{+} indicates that this directory should be populated with the
required files for the constraints declared in file in the
\texttt{model/configuration} directory):

\begin{verbatim}
|- mcm
|  |- photolysis-rates_v3.3.1
|  |- peroxy-radicals_v3.3.1
|- model
|  |- configuration
|  |  |- $TESTNAME.fac
|  |  |- environmentVariables.config
|  |  |- mechanism.reac.cmp
|  |  |- mechanism.prod.cmp
|  |  |- mechanism.species.cmp
|  |  |- mechanism.ro2.cmp
|  |  |- model.parameters
|  |  |- outputSpecies.config
|  |  |- outputRates.config
|  |  |- *photolysisConstant.config
|  |  |- *photolysisConstrained.config
|  |  |- solver.parameters
|  |  |- *speciesConstrained.config
|  |  |- *speciesConstant.config
|  |  |- initialConcentrations.config
|  |  `- a .gitignore file containing
|  |
|  |       # Ignore everything in this directory
|  |       *
|  |       # Except the following
|  |       !*.config
|  |       !*.parameters
|  |       !.gitignore
|  `- constraints
|     |- *+environment (1)
|     |  `- a .gitignore file containing
|     |     # Ignore nothing in this directory
|     |
|     |         # Except this file
|     |         !.gitignore
|     |
|     |- *+photolysis (1)
|     |  `- a .gitignore file containing
|     |             # Ignore nothing in this directory
|     |
|     |         # Except this file
|     |         !.gitignore
|     |
|     `- *+species (1)
|        `- a .gitignore file containing
|               # Ignore nothing in this directory
|
|               # Except this file
|               !.gitignore
|- output
|  |- reactionRates/ (3)
|  |- concentration.output.cmp
|  |- environmentVariables.output.cmp
|  |- errors.output.cmp
|  |- finalModelState.output.cmp
|  |- initialConditionsSetting.output.cmp
|  |- jacobian.output.cmp
|  |- lossRates.output.cmp
|  |- mainSolverParameters.output.cmp
|  |- photolysisRates.output.cmp
|  |- photolysisRatesParameters.output.cmp
|  `- productionRates.output.cmp
|- $TESTNAME.out.cmp (2)
\end{verbatim}

Notes on this structure:

\begin{itemize}
\item if any environment variables (resp. species, photolysis) are to
  be constrained by data from a file (as set in
  \texttt{model/configuration/environmentVariables.config},
  \texttt{model/configuration/speciesConstrained.config},\\
  \texttt{model/configuration/photolysisConstrained.config}), the
  subdirectories in \texttt{model/constraints/}
  (\texttt{environment/}, \texttt{species/}, \texttt{photolysis/})
  should contain data files with filename equal to the constrained
  variable name.
\item the file \texttt{\$TESTNAME.out.cmp}, should contain a copy of
  the expected screen output;
\item the subdirectory \texttt{reactionRates}, should contain a
  \texttt{.gitignore} file and a copy of each of the appropriate files
  normally outputted to \texttt{reactionRates}, with each suffixed by
  \texttt{.cmp}. The \texttt{.gitignore} file should contain

\begin{verbatim}
       \# Ignore everything in this folder
       \*
       \# except files ending in .cmp
       !*.cmp
\end{verbatim}
\end{itemize}

New tests will be picked up by the Makefile automatically when running
\texttt{make\ test}.

% -------------------------------------------------------------------- %
\section{Style Guide} \label{sec:style-guide}

In order to make the code more readable, we attempt to use a
consistent style of coding. Two scripts, \texttt{build/fix\_style.py}
and \texttt{build/fix\_indent.py}, help with keeping the style of the
Fortran code consistent:

\begin{itemize}
\item \texttt{build/fix\_style.py} edits files in-place to try to be
  consistent with the style guide (passing two arguments sends the
  output to the second argument, leaving the input file untouched, and
  is thus the safer option). This script is by no means infallible.;
  therefore, when using the script (by invoking \texttt{python\
    build/fix\_style.py\ filename}), it is strongly recommended to
  have a
  backup of the file to revert to, in case this script wrongly edits.\\
  This script is also used in the \hyperref[sec:test-suite]{Test Suite}
  to check a few aspects of the styling. This works by running the
  script over the source file and outputting to a \texttt{.cmp} file:
  if the copy matches the original file, then the test passes.
\item \texttt{build/fix\_indent.py} works similarly, but checks and
  corrects the indentation level of each line of code. This is also
  used within the \hyperref[sec:test-suite]{Test Suite}.
\end{itemize}

\subsection{Style recommendations} \label{subsec:style-recommendations}

\subsubsection{General principles}

\begin{itemize}
\item All code should be within a module structure, except the main
  program. In our case, due to a complicating factor with linking to
  CVODE, we also place \texttt{FCVFUN()} and \texttt{FCVJTIMES()}
  within the main file \texttt{atchem.f90}.
\item Code is write in free-form Fortran, so source files should end
  in \texttt{.f90}
\item Use two spaces to indent blocks
\item Comment each procedure with a high-level explanation of what
  that procedure does.
\item Comment at the top of each file with author, date, purpose of
  code.
\item Anything in comments is not touched by the style guide, although
  common sense rules, and any code within comments should probably
  follow the rules below.
\end{itemize}

\subsubsection{Specific recommendations}

\begin{itemize}
\item All \textbf{keywords} are lowercase, e.g., \texttt{if\ then},
  \texttt{call}, \texttt{module}, \texttt{integer}, \texttt{real},
  \texttt{only}, \texttt{intrinsic}. This also includes
  \texttt{(kind=XX)} and \texttt{(len=XX)} statements.
\item All \textbf{intrinsic} function names are lowercase, e.g.,
  \texttt{trim}, \texttt{adjustl}, \texttt{adjustr}.
\item \textbf{Relational operators} should use
  \texttt{$\geq$}, \texttt{==} rather than \texttt{.GE.},
  \texttt{.EQ.}, and surrounded by a single space.
\item \texttt{=} should be surrounded by one space when used as
  assignment, except in the cases of \texttt{(kind=XX)} and
  \texttt{(len=XX)} where no spaces should be used.
\item \textbf{Mathematical operators} should be surrounded by one
  space, e.g., \texttt{*}, \texttt{-}, \texttt{+}, \texttt{**}.
  \begin{itemize}
  \item The case of scientific number notation requires no spaces
    around the \texttt{+} or \texttt{-}, e.g., \texttt{1.5e-9}.
  \end{itemize}
\item \textbf{Variables} begin with lowercase, while
  \textbf{procedures} (that is, subroutines and functions) begin with
  uppercase. An exception is \textbf{third-party functions}, which
  should be uppercase. Use either CamelCase or underscores to write
  multiple-word identifiers.
\item \textbf{All procedures and modules} should include the `implicit
  none' statement.
\item All variable \textbf{declarations} should include the
  \texttt{::} notation.
\item All procedure dummy arguments should include an \textbf{intent}
  statement in their declaration.
\item \textbf{Brackets}:
  \begin{itemize}
  \item Opening brackets always have no space before them, except for
    \texttt{read}, \texttt{write}, \texttt{open}, \texttt{close}
    statements.
  \item \texttt{call} statements, and the definitions of all
    procedures should contain \textbf{one} space inside the brackets
    before the first argument and after the last argument, e.g.,
    \texttt{call\ function\_name(\ arg1,\ arg2\ )},
    \texttt{subroutine\ subroutine\_name(\ arg1\ )}
  \item Functions calls, and array indices have \textbf{no such space}
    before the first argument or after the last argument.
  \end{itemize}
\end{itemize}
