\chapter{Test Suite}

AtChem2 uses \href{https://travis-ci.org/}{Travis CI} for Continuous
Integration testing. This programming approach ensures changes to the
code do not modify the behaviour and the results of the software in an
unintended fashion.

To begin using CI on code modifications, create a Pull Request on github
from your own fork to \texttt{AtChem/AtChem2} (see {[}{[}3. Model
Development{]}{]} for instructions on how to set up \textbf{git}). Once
the PR is created, Travis CI will automatically run build, unit and
behaviour tests on 2 architectures (linux and OSX). Pull requests should
only be merged once the Travis CI has completed with passes on both
architectures. This is indicated by the meassage: ``All checks have
passed''.

In order to run the Testsuite on your local machine, call
\texttt{make\ alltest} from the \emph{main directory}. This will run
each of the 3 classes of test in this order: * unit tests: checks that
small fragments of code generate the expected outputs; * build test:
checks that an example program builds and runs successfully; * behaviour
tests: builds each of a number of test setups in turn, and checks that
they generate the expected outputs.

Each of the test classes outputs the results of their tests to the
terminal screen. To perform just the unit tests, call
\texttt{make\ unittests}. To run just the build and behaviour tests,
call \texttt{make\ tests}.

\begin{center}\rule{0.5\linewidth}{\linethickness}\end{center}

The CI tester performs the following on each architecture: * Install
\texttt{gfortran}, \texttt{cvode}, and \texttt{numdiff} * linux: use
\texttt{apt-get} for \texttt{gfortran}, \texttt{numdiff}, and
\texttt{liplapack-dev} (a dependency of \texttt{cvode}). Install
\texttt{cvode} from source (\texttt{apt-get} could also be used to
install \texttt{sundials} (including \texttt{cvode}), but it doesn't
currently hold \texttt{cvode\ 2.9}). * OSX: use Homebrew for
\texttt{gfortran} and \texttt{numdiff}. Install \texttt{cvode} from
source. * Build and run unit tests. PASS if all unit tests pass. * Build
and run a single example of AtChem2. PASS if this exits with 0. * Build
and run several other examples of AtChem2, using different input files.
PASS if no differences from the reference output files are found,
otherwise FAIL. Every test must pass to allow the full CI to PASS.

\hypertarget{adding-new-unit-tests}{%
\subsection{Adding new unit tests}\label{adding-new-unit-tests}}

To add new unit tests, do the following: 1. Navigate to
\texttt{travis/unit\_tests}. This contains several files with the ending
\texttt{*\_test.f90}. IF the new test to be added fits into an existing
test file, edit that file - otherwise, make a new file, but it must
follow that pattern of \texttt{*\_test.f90}. It is suggested that unit
tests covering functions from the source file \texttt{xFunctions.f90}
should be named \texttt{x\_test.f90}. 1. The file must contain a module
with the same name as the file, i.e.~\texttt{*\_test}. It must
\texttt{use\ fruit}, and any other modules as needed. 1. The module
should contain subroutines with the naming scheme
\texttt{test\_*\textasciitilde{}.\ These\ subroutines\ must\ take\ no\ arguments\ (and,\ crucially,\ not\ have\ any\ brackets\ for\ arguments\ either\ -}subroutine
test\_calc\texttt{is\ correct,\ but}subroutine
test\_calc()\texttt{is\ wrong).\ \ 1.\ Each\ subroutine\ should\ call\ one\ or\ more\ assert\ functions\ (usually}assert\_equals()\texttt{,}assert\_not\_equals()\texttt{,}assert\_true()\texttt{or}assert\_false()`).
These assert functions act as the arbiters of pass or failure - each
assert must pass for the subroutine to pass, and each subroutine must
pass for the unit tests to pass. 1. The assert functions have the
following syntax:

\begin{verbatim}
call assert_true( a == b , "Test that a and b are equal")
call assert_false( a == b , "Test that a and b are not equal")
call assert_equals( a, b , "Test that a and b are equal")
call assert_not_equals( a, b , "Test that a and b are not equal")
\end{verbatim}

It is useful to use the last argument as a \emph{unique} and
\emph{descriptive} test message. If any unit tests fail, then this will
be highlighted in the summary, and the message will be printed. Unique
and descriptive messages enable faster and easier understanding of which
test has failed, and perhaps why.

If these steps are followed, calling \texttt{make\ unittests} is enough
to run all the unit tests, including new ones. To check that your new
tests have indeed been run and passed, check the output summary - you
should see a line associated to each of the \texttt{test*} subroutines
in each file in the unit test suite.

\hypertarget{adding-new-behaviour-tests}{%
\subsection{Adding new behaviour
tests}\label{adding-new-behaviour-tests}}

To add a new behaviour test called `\$TESTNAME' to the Testsuite, you
should provide the following:

Each input
\(TESTNAME should have a subdirectory `travis/tests/\)TESTNAME/\texttt{containing\ the\ following\ files\ in\ the\ following\ structure\ (}*\texttt{indicates\ that\ this\ file/directory\ is\ optional\ dependent\ on\ the\ configuration\ used\ in\ the\ test,\ while}+\texttt{indicates\ that\ this\ directory\ should\ be\ populated\ with\ the\ required\ files\ for\ the\ constraints\ declared\ in\ file\ in\ the}model/configuration`
directory):

\begin{verbatim}
|- mcm
|  |- photolysis-rates_v3.3.1
|  |- peroxy-radicals_v3.3.1
|- model
|  |- configuration
|  |  |- $TESTNAME.fac
|  |  |- environmentVariables.config
|  |  |- mechanism.reac.cmp
|  |  |- mechanism.prod.cmp
|  |  |- mechanism.species.cmp
|  |  |- mechanism.ro2.cmp
|  |  |- model.parameters
|  |  |- outputSpecies.config
|  |  |- outputRates.config
|  |  |- *photolysisConstant.config
|  |  |- *photolysisConstrained.config
|  |  |- solver.parameters
|  |  |- *speciesConstrained.config
|  |  |- *speciesConstant.config
|  |  |- initialConcentrations.config
|  |  `- a .gitignore file containing 
|  |
|  |       # Ignore everything in this directory
|  |       *
|  |       # Except the following
|  |       !*.config
|  |       !*.parameters
|  |       !.gitignore
|  `- constraints
|     |- *+environment (1)
|     |  `- a .gitignore file containing
|     |     # Ignore nothing in this directory
|     |
|     |         # Except this file
|     |         !.gitignore
|     |
|     |- *+photolysis (1)
|     |  `- a .gitignore file containing
|     |             # Ignore nothing in this directory
|     |
|     |         # Except this file
|     |         !.gitignore
|     |
|     `- *+species (1)
|        `- a .gitignore file containing
|               # Ignore nothing in this directory
|
|               # Except this file
|               !.gitignore
|- output
|  |- reactionRates/ (3)
|  |- concentration.output.cmp
|  |- environmentVariables.output.cmp
|  |- errors.output.cmp
|  |- finalModelState.output.cmp
|  |- initialConditionsSetting.output.cmp
|  |- jacobian.output.cmp
|  |- lossRates.output.cmp
|  |- mainSolverParameters.output.cmp
|  |- photolysisRates.output.cmp
|  |- photolysisRatesParameters.output.cmp
|  `- productionRates.output.cmp
|- $TESTNAME.out.cmp (2)
\end{verbatim}

Notes on this structure: 1. if any environment variables (resp. species,
photolysis) are to be constrained by data from a file (as set in
\texttt{model/configuration/environmentVariables.config},
\texttt{model/configuration/speciesConstrained.config},
\texttt{model/configuration/photolysisConstrained.config}), the
subdirectories in \texttt{model/constraints/} (\texttt{environment/},
\texttt{species/}, \texttt{photolysis/}) should contain data files with
filename equal to the constrained variable name. 1. the file
\texttt{\$TESTNAME.out.cmp}, should contain a copy of the expected
screen output; 1. the subdirectory \texttt{reactionRates}, should
contain a \texttt{.gitignore} file and a copy of each of the appropriate
files normally outputted to \texttt{reactionRates}, with each suffixed
by \texttt{.cmp}. The \texttt{.gitignore} file should contain

\begin{verbatim}
       \# Ignore everything in this folder
       \*
       \# except files ending in .cmp
       !*.cmp
\end{verbatim}

New tests will be picked up by the Makefile automatically when running
\texttt{make\ test}.
