Two versions of Atchem2 are available:

\begin{enumerate}
\def\labelenumi{\arabic{enumi})}
\item
  the stable version, which is indicated by a version number (e.g.,
  \textbf{v1.0}), and can be found
  \href{https://github.com/AtChem/AtChem2/releases}{here}.
\item
  the development version: which is indicated by a version number with
  the suffix \texttt{-dev} (e.g., \textbf{v1.1-dev}), and can be
  downloaded from the \texttt{master\ branch}
  (https://github.com/AtChem/AtChem2/archive/master.zip) or obtained via
  \textbf{git}.
\end{enumerate}

AtChem2 is under active development, which means that the
\texttt{master\ branch} may sometimes be a few steps ahead of the latest
stable release. The {[}{[}test suite\textbar{}3.1 Test Suite{]}{]} is
designed to ensure that changes to the code do not cause unintended
behaviour or unexplained differences in the model results, so the
development version is usually safe to use, although caution is advised.

The roadmap for the development of Atchem2 can be found
\href{https://github.com/AtChem/AtChem2/projects/1}{here}.

Feedback, bug reports, comments and suggestions are welcome. Please
check \href{https://github.com/AtChem/AtChem2/issues}{this page} for a
list of known and current issues.

\begin{center}\rule{0.5\linewidth}{\linethickness}\end{center}

If you want to contribute to the model development, the best way is to
use \textbf{git}. The procedure to contribute code is described below. A
basic level of
\href{https://swcarpentry.github.io/git-novice/}{knowledge of git} is
\emph{required}.

\begin{enumerate}
\def\labelenumi{\arabic{enumi}.}
\item
  Fork the official repository (\texttt{AtChem/AtChem2}) to your github
  account (\texttt{username/AtChem2}).
\item
  Configure git so that \texttt{origin} is your fork
  (\texttt{username/AtChem2}) and \texttt{upstream} is the official
  repository (\texttt{AtChem/AtChem2}). The output of
  \texttt{git\ remote\ -v} should look like this:
  \texttt{origin\ \ git@github.com:username/AtChem2.git\ (fetch)\ \ origin\ \ git@github.com:username/AtChem2.git\ (push)\ \ upstream\ \ \ \ git@github.com:AtChem/AtChem2.git\ (fetch)\ \ upstream\ \ \ \ git@github.com:AtChem/AtChem2.git\ (push)}
\item
  Create a new branch in your local repository. Make your edits on the
  branch, commit and push. Before committing, it is advised to run the
  {[}{[}test suite\textbar{}3.1 Test Suite{]}{]} locally to verify
  whether the changes could cause any problem.
\item
  Submit a pull request, together with a brief description of the
  proposed changes. One of the admins will review the edits and approve
  them or ask for additional changes, as appropriate.
\end{enumerate}

Contributions can also be submitted via email or via the
\href{https://github.com/AtChem/AtChem2/issues}{issues page}.

A {[}{[}Style Guide\textbar{}3.2 Style Guide{]}{]} is available for code
contributions. Note that style and indentation of the code are also
checked by the {[}{[}test suite\textbar{}3.1 Test Suite{]}{]}.
