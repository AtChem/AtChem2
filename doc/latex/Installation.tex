% -----------------------------------------------------------------------------
%
% Copyright (c) 2017 Sam Cox, Roberto Sommariva
%
% This file is part of the AtChem2 software package.
%
% This file is covered by the MIT license which can be found in the
% file LICENSE.md at the top level of the AtChem2 distribution.
%
% -----------------------------------------------------------------------------
\chapter{Model Installation} \label{ch:installation}

% -------------------------------------------------------------------- %
\section{Requirements} \label{sec:requirements}

AtChem2 can be installed on Linux/Unix or macOS. A working knowledge
of the \textbf{unix shell} and its
\href{https://swcarpentry.github.io/shell-novice/reference/}{basic
  commands} is \emph{required} to install and use the model.

AtChem2 needs the following tools:

\begin{itemize}
\item a Fortran compiler: the model compiles with GNU
  \texttt{gfortran} (version 4.8.5) and with Intel \texttt{ifort}
  (version 17.0)
\item Python 2.7.x
\item cmake
\item Ruby 2.0 (optional)
\end{itemize}

Some or all of these tools may already be present on your system. Use
the \texttt{which} command to find out (e.g., \texttt{which\ python},
\texttt{which\ cmake}, etc\ldots{}). Otherwise, check the local
documentation or ask the system administrator.

In addition, AtChem2 has the following dependencies:

\begin{itemize}
\item the CVODE library
\item the openlibm library
\item the BLAS and LAPACK libraries
\item numdiff (optional)
\item FRUIT (optional)
\end{itemize}

For detailed instructions on the installation and configuration of the
dependencies go to: \hyperref[sec:dependencies]{Dependencies}.

% -------------------------------------------------------------------- %
\section{Download} \label{sec:download}

There are two versions of AtChem2: the
\href{https://github.com/AtChem/AtChem2/releases}{stable version} and
the development version, also known as \texttt{master\ branch} (see
the \hyperref[ch:development]{Model Development} section for
additional information). The source code can be obtained in two ways:

\begin{itemize}
\item with \textbf{git}:
  \begin{enumerate}
  \item Open the terminal. Move to the directory where you want to
    install AtChem2.
  \item Execute \texttt{git\ clone\
      https://github.com/AtChem/AtChem2.git} (if using HTTPS) or
    \texttt{git\ clone\ git@github.com:AtChem/AtChem2.git} (if using
    SSH). This method will download the development version and it is
    recommended if you want to contribute to the model development.
  \end{enumerate}
\item with the \textbf{archive file} (\texttt{*.tar.gz} or
  \texttt{*.zip}):
  \begin{enumerate}
  \item Download the archive file of the
    \href{https://github.com/AtChem/AtChem2/releases}{stable version}
    or of the
    \href{https://github.com/AtChem/AtChem2/archive/master.zip}{development
      version} to the directory where you want to install AtChem2.
  \item Open the terminal. Move to the directory where you downloaded
    the archive file.
  \item Unpack the archive file (e.g., \texttt{tar\ -zxvf\
      v1.1.tar.gz} or \texttt{unzip\ master.zip}).
  \end{enumerate}
\end{itemize}

Depending on which of these methods you have used, the source code is
now in a directory called \texttt{AtChem2} or \texttt{AtChem2-1.1} or
\texttt{AtChem2-master}. This directory -- which you can rename, if
you want to -- is the \emph{main directory} of the model. In the
documentation we will assume that the \emph{AtChem2 main directory} is
\texttt{\$HOME/AtChem2}.

% -------------------------------------------------------------------- %
\section{Dependencies} \label{sec:dependencies}

AtChem2 has a number of dependencies (external tools and libraries):
some are required and without them the model cannot be installed or
used, others are optional. We suggest to use a single directory for
all the dependencies; the \emph{dependencies directory} can be located
anywhere and called as you prefer. In the documentation, we will
assume that the \emph{dependencies directory} is
\texttt{\$HOME/atchem-libraries/}.

Before installing the dependencies, make sure that Fortran, Python,
cmake and (optionally) Ruby are installed on your system, as explained
in the \hyperref[sec:install]{Installation} section.

\subsection{Required dependencies} \label{subsec:required-dependencies}

\subsubsection{BLAS and LAPACK} \label{blas-and-lapack}

BLAS and LAPACK are standard Fortran libraries for linear
algebra. They are needed to install and compile the CVODE library (see
below). Usually, they are in \texttt{/usr/lib/} (e.g.,
\texttt{/usr/lib/libblas/} and \texttt{/usr/lib/lapack/}). The
location may be different, especially if you are on an HPC system, so
check the local documentation or ask the system administrator.

\subsubsection{CVODE} \label{cvode}

AtChem2 uses the CVODE library, which is part of the
\href{https://computation.llnl.gov/projects/sundials}{SUNDIALS} suite,
to solve the system of ordinary differential equation (ODE). The
current version of CVODE is 2.9.0 (part of SUNDIALS 2.7.0) and can be
installed using the \texttt{install\_cvode.sh} script in the
\texttt{tools/install/} directory.

\begin{enumerate}
\item Move to the \emph{AtChem2 main directory} (e.g., \texttt{cd\
    \textasciitilde{}/AtChem2}).
\item Open the installation script
  (\texttt{tools/install/install\_cvode.sh}) with a text editor:
  \begin{itemize}
  \item If LAPACK and BLAS are not in the default location on your
    system (see above), change the \texttt{LAPACK\_LIBS} variable for
    your architecture (Linux or macOS), as appropriate.
  \item If you are not using the \texttt{gcc} compiler
    (\texttt{gfortran} is part of \texttt{gcc}), change the line
    \texttt{-DCMAKE\_C\_COMPILER:FILEPATH=gcc\ \textbackslash{}}
    accordingly.
  \end{itemize}
\item From the \emph{AtChem2 main directory}, run the installation
  script (change the path of the \emph{dependencies directory} as
  needed):
  \begin{verbatim}
  ./tools/install/install_cvode.sh ~/atchem-libraries/
  \end{verbatim}
\end{enumerate}

If the installation is successful, there should be a working CVODE
installation at
\texttt{\textasciitilde{}/atchem-libraries/cvode/}. The path to the
CVODE library is
\texttt{\textasciitilde{}/atchem-libraries/cvode/lib/}.

\subsubsection{openlibm} \label{openlibm}

openlibm is a \href{http://openlibm.org/}{portable version} of the
\href{https://en.wikipedia.org/wiki/C_mathematical_functions}{libm}
library. Installing this library and linking against it allows
reproducible results by ensuring the same implementation of several
mathematical functions across platforms.

The current version of openlibm is 0.4.1 and can be installed using
the \texttt{install\_openlibm.sh} script in the
\texttt{tools/install/} directory.

\begin{enumerate}
\item Move to the \emph{AtChem2 main directory} (e.g., \texttt{cd\
    \textasciitilde{}/AtChem2}).
\item Run the installation script (change the path of the
  \emph{dependencies directory} as needed):
  \begin{verbatim}
  ./tools/install/install_openlibm.sh ~/atchem-libraries/
  \end{verbatim}
\end{enumerate}

If the installation is successful, there should be a working openlibm
installation at
\texttt{\textasciitilde{}/atchem-libraries/openlibm-0.4.1/}.

\subsection{Optional dependencies} \label{subsec:optional-dependencies}

\subsubsection{numdiff} \label{numdiff}

numdiff is a \href{https://www.nongnu.org/numdiff/}{program} used to
compare files containing numerical fields. It is needed only if you
want to run the \hyperref[sec:testsuite]{Test Suite}, a series of
tests to ensure that the model works properly. Installation of numdiff
is recommended if you want to contribute to the development of
AtChem2.

Use \texttt{which\ numdiff} to check if the program is already
installed on your system. If not, you can install it locally, for
example in the \emph{dependencies directory}. Use the script
\texttt{install\_numdiff.sh} in the \texttt{tools/install/} directory.

\begin{enumerate}
\item Move to the \emph{AtChem2 main directory} (e.g., \texttt{cd\
    \textasciitilde{}/AtChem2}).
\item Run the installation script (change the path of the
  \emph{dependencies directory} as needed):
  \begin{verbatim}
  ./tools/install/install_numdiff.sh ~/atchem-libraries/
  \end{verbatim}
\item Move to your \texttt{\$HOME} directory (\texttt{cd\
    \textasciitilde{}}). Open the \texttt{.bash\_profile} file (or the
  \texttt{.profile} file, depending on your configuration) with a text
  editor. Add the following line at the bottom of the file (change the
  path of the \emph{dependencies directory} as needed):
  \texttt{PATH=\$PATH:\$HOME/atchem-libraries/numdiff/bin}
\item Close the terminal.
\item Open the terminal and execute \texttt{which\ numdiff} to check
  that the program has been installed correctly.
\end{enumerate}

\subsubsection{FRUIT} \label{fruit}

FRUIT (FORTRAN Unit Test Framework) is a
\href{https://en.wikipedia.org/wiki/Unit_testing}{unit test framework}
for Fortran. It requires Ruby 2.0 and is needed only if you want to
run the unit tests in the \hyperref[sec:testsuite]{Test Suite}.
Installation of FRUIT is recommended if you want to contribute to the
development of AtChem2.

The current version of FRUIT is 3.4.3 and can be installed using the
\texttt{install\_fruit.sh} script in the \texttt{tools/install/}
directory.

\begin{enumerate}
\item Move to the \emph{AtChem2 main directory} (e.g., \texttt{cd\
    \textasciitilde{}/AtChem2}).
\item Run the installation script (change the path of the
  \emph{dependencies directory} as needed):
  \begin{verbatim}
  ./tools/install/install_fruit.sh ~/atchem-libraries/
  \end{verbatim}
\end{enumerate}

If the installation is successful, there should be a working FRUIT
installation at
\texttt{\textasciitilde{}/atchem-libraries/fruit\_3.4.3/}.

% -------------------------------------------------------------------- %
\section{Install} \label{sec:install}

To install AtChem2:

\begin{enumerate}
\item Move to the \emph{AtChem2 main directory} (\texttt{cd\
    \textasciitilde{}/AtChem2/}). Install the
  \hyperref[sec:dependencies]{Dependencies} and take note of the name
  and path of the \emph{dependencies directory} (in the following
  instructions, we will assume that the \emph{dependencies directory}
  is \texttt{\textasciitilde{}/atchem-libraries/}).
\item Copy the \texttt{Makefile} in the \texttt{tools/} directory to
  the \emph{main directory} (\texttt{cp\ tools/Makefile\ ./}).
\item From the the \emph{main directory}, open the \texttt{Makefile}
  with a text editor. Set the variables \texttt{CVODELIB},
  \texttt{OPENLIBMDIR}, \texttt{FRUITDIR} to the paths of the CVODE,
  openlibm and FRUIT libraries, as described in the
  \hyperref[sec:dependencies]{Dependencies} section. Use the full path
  to the libraries, not the relative path (see issue
  \href{https://github.com/AtChem/AtChem2/issues/364}{\#364}). For
  example:
  \begin{verbatim}
  CVODELIB = $(HOME)/atchem-libraries/cvode/lib
  OPENLIBMDIR = $(HOME)/atchem-libraries/openlibm-0.4.1
  FRUITDIR = $(HOME)/atchem-libraries/fruit_3.4.3
  \end{verbatim}
\item Execute \texttt{./build/build\_atchem2.sh\
    ./mcm/mechanism\_test.fac}. This command compiles the model and
  creates an executable (\texttt{atchem2}) using the test mechanism
  file \texttt{mechanism\_test.fac} in the \texttt{mcm/} directory.
\item Execute \texttt{./atchem2}. If the model has been installed
  correctly, you should see a message similar to this:
  \begin{verbatim}
  ------------------
   Final statistics
  ------------------
   No. steps = 546   No. f-s = 584   No. J-s = 912   No. LU-s = 56
   No. nonlinear iterations = 581
   No. nonlinear convergence failures = 0
   No. error test failures = 4


   Runtime = 0
   Deallocating memory.
  \end{verbatim}
\end{enumerate}

This means that AtChem2 has completed the test run without errors and
is ready to be used. The directory structure of AtChem2 is described
in \hyperref[sec:structure]{Model Structure}. For instructions on how
to set up, compile and execute the model go to:
\hyperref[ch:setup]{Model Setup} and \hyperref[ch:execution]{Model
  Execution}.

% -------------------------------------------------------------------- %
\section{Tests (optional)} \label{sec:tests}

You can run the \hyperref[sec:testsuite]{Test Suite} to verify that
AtChem2 has been installed properly and to make sure that changes to
the code do not result in unintended behaviour. This is recommended if
you want to contribute to the model development. Note that running the
Test Suite requires the optional dependencies to be installed, as
explained in the \hyperref[sec:dependencies]{Dependencies} section.

To run the tests, execute the following commands from the
\emph{AtChem2 main directory}:

\begin{itemize}
\item \texttt{make\ alltests} runs all the tests (requires
  \textbf{numdiff} and \textbf{FRUIT}).
\item \texttt{make\ tests} runs only the behaviour tests (requires
  \textbf{numdiff}).
\item \texttt{make\ unittests} runs only the unit tests (requires
  \textbf{FRUIT}).
\end{itemize}

For more information on the Test Suite go to the corresponding
\hyperref[sec:testsuite]{section}.

% -------------------------------------------------------------------- %
\section{Model Structure} \label{sec:structure}

AtChem2 is organized in several directories containing the source
code, the compilation files, the chemical mechanism, the model
configuration and output files, a number of scripts to install and
compile the model, plotting tools in various programming languages,
and the test suite files.

The directory structure has changed with the release of
\textbf{version 1.1} (November 2018). The following table shows the
new structure and, for reference, the previous one:

{\footnotesize
  \begin{tabular}{llp{5cm}}
    v1.0 & v1.1 & description\\
    \hline
    main directory & main directory & information files (changelog, citation, license, readme) and auxiliary files for the test suite (N.B.: the .gcda and .gcno files are generated by the Fortran compiler during the build process).\\
    -- & \texttt{mcm/} & data files related to specific versions of the MCM: lists of organic peroxy radicals (RO2), parameters to calculate photolysis rates, example mechanism file.\\
    -- & \texttt{model/} & model files: chemical mechanism (.fac), configuration, input, output.\\
    \texttt{modelConfiguration/} & \texttt{model/configuration/} & model configuration files.\\
    -- & \texttt{model/constraints/} & model constraints.\\
    \texttt{environmentConstraints/} & \texttt{model/constraints/environment/} & constrained environment variables.\\
    \texttt{environmentConstraints/} & \texttt{model/constraints/photolysis/} & constrained photolysis rates.\\
    \texttt{speciesConstraints/} & \texttt{model/constraints/species/} & constrained chemical species.\\
    \texttt{modelOutput/} & \texttt{model/output/} & model output: chemical species, environment variables and photolysis rates, diagnostic variables, formatted production and loss rates of selected species.\\
    \texttt{instantaneousRates/} & \texttt{model/output/reactionRates/} & model output: reaction rates of every reaction in the chemical mechanism.\\
    \texttt{obj/} & \texttt{obj/} & files generated by the Fortran compiler.\\
    \texttt{src/} & \texttt{src/} & Fortran source files.\\
    -- & \texttt{src/gen/} & Fortran source files generated by the compiler from the chemical mechanism.\\
    \texttt{tools/} & \texttt{tools/} & Python and shell scripts to install, build and compile AtChem2.\\
    -- & \texttt{build/} & Python and shell scripts to build and compile AtChem2.\\
    \texttt{tools/install/} & \texttt{tools/install/} & shell scripts to install the dependencies.\\
    -- & \texttt{tools/plot/} & scripts to plot the model results (gnuplot, Matlab/Octave, Python, R).\\
    \texttt{travis/} & \texttt{travis/} & shell scripts to run the test suite.\\
    \texttt{travis/tests/} & \texttt{travis/tests/} & behaviour tests.\\
    -- & \texttt{travis/unit\_tests/} & unit tests.\\
  \end{tabular}
}

The \texttt{model/} directory is the most important for the user: it
includes the chemical mechanism, the configuration files, the model
constraints and the model output. The \texttt{model/} directory can be
given any name and it can also be located outside of the \emph{AtChem2
  main directory}.

There can be multiple \texttt{model/} directories (with different
names) in the same location. As long as the correct paths are passed
to the compilation and execution scripts, the model will compile and
run. This approach gives the user the flexibility to run different
versions of the same model or different models at the same time. For
more information go to: \hyperref[ch:execution]{Model Execution}.
