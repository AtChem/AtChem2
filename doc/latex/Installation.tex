% -----------------------------------------------------------------------------
%
% Copyright (c) 2017 Sam Cox, Roberto Sommariva
%
% This file is part of the AtChem2 software package.
%
% This file is covered by the MIT license which can be found in the file
% LICENSE.md at the top level of the AtChem2 distribution.
%
% -----------------------------------------------------------------------------

\chapter{Model Installation} \label{ch:installation}

% -------------------------------------------------------------------- %
\section{Requirements} \label{sec:requirements}

AtChem2 can be installed on Linux/Unix or macOS operating systems. A
working knowledge of the \textbf{unix shell} and its
\href{https://swcarpentry.github.io/shell-novice/reference/}{basic commands}
is \emph{required} to install and use the model. AtChem2 requires the
following tools:

\begin{itemize}
\item Fortran: the model compiles with GNU \texttt{gfortran} (version
  4.8.5) and with Intel \texttt{ifort} (version 17.0)
\item Python
\item cmake
\item Ruby (optional)
\end{itemize}

Some or all of these tools may already be present on your system. Use
the \texttt{which} command to find out (e.g.: \verb|which python|,
\verb|which cmake|, etc\ldots). Otherwise, check the local
documentation or ask the system administrator. In addition, AtChem2
has the following dependencies:

\begin{itemize}
\item CVODE
\item openlibm
\item BLAS and LAPACK
\item numdiff (optional)
\item FRUIT (optional)
\end{itemize}

For detailed instructions on the installation and configuration of the
dependencies go to Sect.~\ref{sec:dependencies}.

% -------------------------------------------------------------------- %
\section{Download} \label{sec:download}

There are two versions of AtChem2: the
\href{https://github.com/AtChem/AtChem2/releases}{stable version} and
the development version, also known as \texttt{master\ branch} (see
Chapt.~\ref{ch:development} for additional information). The source
code can be obtained in two ways:

\begin{itemize}
\item with \textbf{git}:
  \begin{enumerate}
  \item Open the terminal. Move to the directory where you want to
    install AtChem2.
  \item Execute \verb|git clone https://github.com/AtChem/AtChem2.git|
    (if using HTTPS) or \verb|git clone git@github.com:AtChem/AtChem2.git|
    (if using SSH).  This method will download the development version
    and it is recommended if you want to contribute to the model
    development.
  \end{enumerate}
\item with the \textbf{archive file}:
  \begin{enumerate}
  \item Download the archive file (\texttt{*.tar.gz} or \texttt{*.zip}) of the
    \href{https://github.com/AtChem/AtChem2/releases}{stable version} or of the
    \href{https://github.com/AtChem/AtChem2/archive/master.zip}{development version}
    to the directory where you want to install AtChem2.
  \item Open the terminal. Move to the directory where you downloaded
    the archive file.
  \item Execute \verb|tar -zxfv v1.2.tar.gz| or
    \verb|unzip -v master.zip| to unpack the archive file.
  \end{enumerate}
\end{itemize}

The downloaded source code is now in a directory called
\texttt{AtChem2} (if git was used) or \texttt{AtChem2-1.2} (if the
stable version was downloaded) or \texttt{AtChem2-master} (if the
development version was downloaded). This directory, which can be
renamed as you prefer, is the \maindir\ of the model. In this manual
we will assume that the \maindir\ is \texttt{\$HOME/AtChem2}.

% -------------------------------------------------------------------- %
\section{Dependencies} \label{sec:dependencies}

AtChem2 has a number of dependencies (external tools and libraries):
some are required, and without them the model cannot be installed or
used, others are optional. We suggest to use a single directory for
all the dependencies; the \emph{dependencies directory} can be located
anywhere and called as you prefer. In the documentation, we will
assume that the \emph{dependencies directory} is
\texttt{\$HOME/atchem-lib/}.

Before installing the dependencies, make sure that a \textbf{Fortran}
compiler, \textbf{Python}, \textbf{cmake} and (optionally)
\textbf{Ruby} are installed on your system, as explained in
Sect.~\ref{sec:requirements}.

\subsection{Required dependencies} \label{subsec:required-dependencies}

\subsubsection{BLAS and LAPACK}

BLAS and LAPACK are standard Fortran libraries for linear
algebra. They are needed to install and compile the CVODE library (see
below). Usually they are located in the \texttt{/usr/lib/} directory
(e.g., \texttt{/usr/lib/libblas/} and \texttt{/usr/lib/lapack/}). The
location may be different, especially if you are on a High Performance
Computing (HPC) system, so check the local documentation or ask the
system administrator.

\subsubsection{CVODE}

AtChem2 uses the CVODE library, which is part of the open source
\href{https://computation.llnl.gov/projects/sundials}{SUNDIALS} suite,
to solve the system of ordinary differential equation (ODE).

The current version of CVODE is 2.9.0 (part of SUNDIALS 2.7.0) and can
be installed using the \texttt{install\_cvode.sh} script in the
\texttt{tools/install/} directory.

\begin{enumerate}
\item Open the terminal. Move to the \maindir\ (\verb|cd ~/AtChem2|).
\item Open the installation script
  (\texttt{tools/install/install\_cvode.sh}) with a text editor:
  \begin{itemize}
  \item If LAPACK and BLAS are not in the default location on your
    system (see above), change the \texttt{LAPACK\_LIBS} variable for
    your architecture (Linux or macOS) as appropriate.
  \item If you are not using the \texttt{gcc} compiler
    (\texttt{gfortran}), change the line
    \texttt{-DCMAKE\_C\_COMPILER:FILEPATH=gcc \textbackslash}
    accordingly.
  \end{itemize}
\item Run the installation script (change the path to the \depdir\ as
  needed):
  \begin{verbatim}
  ./tools/install/install_cvode.sh ~/atchem-lib/
  \end{verbatim}
\end{enumerate}

If the installation is successful, there should be a working CVODE
installation at
\texttt{\textasciitilde{}/atchem-lib/cvode/}. The path to the
CVODE library is
\texttt{\textasciitilde{}/atchem-lib/cvode/lib/}.

\subsubsection{openlibm}

openlibm is a \href{https://openlibm.org/}{portable version} of the
open source libm library. Installing openlibm and linking against it
allows reproducible results by ensuring the same implementation of
several mathematical functions across platforms.

The current version of openlibm is 0.4.1 and can be installed using
the \texttt{install\_openlibm.sh} script in the
\texttt{tools/install/} directory.

\begin{enumerate}
\item Open the terminal. Move to the \maindir\ (\verb|cd ~/AtChem2|).
\item Run the installation script (change the path to the \depdir\ as
  needed):
  \begin{verbatim}
  ./tools/install/install_openlibm.sh ~/atchem-lib/
  \end{verbatim}
\end{enumerate}

If the installation is successful, there should be a working openlibm
installation at
\texttt{\textasciitilde{}/atchem-lib/openlibm-0.4.1/}.

\subsection{Optional dependencies} \label{subsec:optional-dependencies}

\subsubsection{numdiff}

numdiff is a \href{https://www.nongnu.org/numdiff/}{program} used to
compare files containing numerical fields. It is needed only if you
want to run the \hyperref[sec:test-suite]{Test Suite}, a series of
tests used to ensure that the model works properly and that changes to
the code do not result in unintended behaviour. Installation of
numdiff is recommended if you want to contribute to the development of
AtChem2.

Use \verb|which numdiff| to check if the program is already installed
on your system. If not, ask the system administrator. Alternatively,
numdiff can be installed locally (e.g., in the \depdir) using the
script \texttt{install\_numdiff.sh} in the \texttt{tools/install/}
directory.

\begin{enumerate}
\item Open the terminal. Move to the \maindir\ (\verb|cd ~/AtChem2|).
\item Run the installation script (change the path to the \depdir\ as
  needed):
  \begin{verbatim}
  ./tools/install/install_numdiff.sh ~/atchem-lib/
  \end{verbatim}
\item Move to your \texttt{\$HOME} directory (\texttt{cd\
    \textasciitilde{}}). Open the \texttt{.bash\_profile} file (or the
  \texttt{.profile} file, depending on your configuration) with a text
  editor. Add the following line at the bottom of the file (change the
  path of the \emph{dependencies directory} as needed):
  \texttt{PATH=\$PATH:\$HOME/atchem-lib/numdiff/bin}
\item Close the terminal.
\item Reopen the terminal and execute \verb|which numdiff| to check
  that the program has been installed correctly.
\end{enumerate}

\subsubsection{FRUIT}

FRUIT (FORTRAN Unit Test Framework) is a
\href{https://en.wikipedia.org/wiki/Unit_testing}{unit test framework}
for Fortran. It requires Ruby and is needed only if you want to run
the unit tests (see Sect.~\ref{sec:test-suite}). Installation of FRUIT
is recommended if you want to contribute to the development of
AtChem2.

The current version of FRUIT is 3.4.3 and can be installed using the
\texttt{install\_fruit.sh} script in the \texttt{tools/install/}
directory.

\begin{enumerate}
\item Open the terminal. Move to the \maindir\ (\verb|cd ~/AtChem2|).
\item Run the installation script (change the path to the \depdir\ as
  needed):
  \begin{verbatim}
  ./tools/install/install_fruit.sh ~/atchem-lib/
  \end{verbatim}
\end{enumerate}

If the installation is successful, there should be a working FRUIT
installation at
\texttt{\textasciitilde{}/atchem-lib/fruit\_3.4.3/}.

% -------------------------------------------------------------------- %
\section{Install} \label{sec:install}

To install AtChem2:

\begin{enumerate}
\item Move to the \emph{AtChem2 main directory} (\texttt{cd\
    \textasciitilde{}/AtChem2/}). Install the
  \hyperref[sec:dependencies]{Dependencies} and take note of the name
  and path of the \emph{dependencies directory} (in the following
  instructions, we will assume that the \emph{dependencies directory}
  is \texttt{\textasciitilde{}/atchem-lib/}).
\item Copy the \texttt{Makefile} in the \texttt{tools/} directory to
  the \emph{main directory} (\texttt{cp\ tools/Makefile\ ./}).
\item From the the \emph{main directory}, open the \texttt{Makefile}
  with a text editor. Set the variables \texttt{CVODELIB},
  \texttt{OPENLIBMDIR}, \texttt{FRUITDIR} to the paths of the CVODE,
  openlibm and FRUIT libraries, as described in the
  \hyperref[sec:dependencies]{Dependencies} section. Use the full path
  to the libraries, not the relative path (see issue
  \href{https://github.com/AtChem/AtChem2/issues/364}{\#364}). For
  example (change the path to the \depdir\ as needed):
  \begin{verbatim}
  CVODELIB = $(HOME)/atchem-lib/cvode/lib
  OPENLIBMDIR = $(HOME)/atchem-lib/openlibm-0.4.1
  FRUITDIR = $(HOME)/atchem-lib/fruit_3.4.3
  \end{verbatim}
\item Compile AtChem2:
  \begin{verbatim}
  ./build/build_atchem2.sh ./mcm/mechanism_test.fac
  \end{verbatim}
  This command compiles the model and creates an executable
  (\texttt{atchem2}) using the example mechanism file
  \texttt{mechanism\_test.fac} in the \texttt{mcm/} directory.
\item Execute \verb|./atchem2|. This command runs the model executable
  using a default configuration. At the end of the run you should see
  a message similar to this:
  \begin{verbatim}
  ------------------
   Final statistics
  ------------------
   No. steps = 546   No. f-s = 584   No. J-s = 912   No. LU-s = 56
   No. nonlinear iterations = 581
   No. nonlinear convergence failures = 0
   No. error test failures = 4


   Runtime = 0
   Deallocating memory.
  \end{verbatim}
\end{enumerate}

If your are installing on a macOS system and receive an error message
concerning \texttt{rpath}, please check the ``Note for macOS users'' on
the \href{https://github.com/AtChem/AtChem2/wiki/How-to-install-AtChem2}{wiki}.

AtChem2 is now ready to be used. Optionally, the Test Suite can be run
to check that the model has been installed correctly (see
Sect.~\ref{subsec:tests-optional}). The directory structure and the
organization of the AtChem2 model are described in
Sect.~\ref{sec:model-structure}. For information on how to compile,
configure and execute an AtChem2 model go to Chapt.~\ref{ch:setup} and
Chapt.~\ref{ch:execution}.

\subsection{Tests (optional)} \label{subsec:tests-optional}

The \hyperref[sec:test-suite]{Test Suite} can be used to verify that
AtChem2 has been installed correctly and works as intended. It is
recommended to run the Test suite if you want to contribute to the
development of the AtChem2 model. Note that in order to run the Test
Suite the \hyperref[subsec:optional-dependencies]{optional dependencies}
need to be installed.

To run the tests, open the terminal and execute one of the following
commands from the \maindir:

\begin{itemize}
\item \verb|make alltests|: run all the tests (requires numdiff and
  FRUIT).
\item \verb|make tests|: run only the build and behaviour tests
  (requires numdiff).
\item \verb|make unittests|: run only the unit tests (requires
  FRUIT).
\end{itemize}

The command executes the requested tests, then prints to the terminal
the tests output and a summary of the results.

% -------------------------------------------------------------------- %
\section{Model Structure} \label{sec:model-structure}

AtChem2 is organized in several directories which contain the source
code, the compilation files, the chemical mechanism, the model
configuration and output, several scripts and utilities, and the Test
Suite. The directory structure of AtChem2 is derived from that of
AtChem-online, but it was substantially changed with the release of
version 1.1 (November 2018). Table~\ref{tab:atchem-structure} shows
the new directory structure and, for reference, the original one.

\begin{table}[htb]
  \centering \footnotesize
  \caption{Directory structure of AtChem2. ``Original'' refers to version 1.0 and earlier;
    ``New'' refers to version 1.1 and later.} \label{tab:atchem-structure}
  \begin{tabular}{llp{3cm}}
    Original & New & Description\\
    \hline
    --                               & \texttt{build/}                         & scripts to compile the model.\\
    \hline
    --                               & \texttt{doc/}                           & user manual and other documentation.\\
    \hline
    --                               & \texttt{mcm/}                           & MCM data files and example chemical mechanism.\\
    \hline
    \texttt{modelConfiguration/}     & \texttt{model/configuration/}           & model and solver configuration files.\\
    \hline
    \texttt{speciesConstraints/}     & \texttt{model/constraints/species/}     & model constraints: chemical species.\\
    \hline
    \texttt{environmentConstraints/} & \texttt{model/constraints/environment/} & model constraints: environment variables.\\
    \hline
    \texttt{environmentConstraints/} & \texttt{model/constraints/photolysis/}  & model constraints: photolysis rates.\\
    \hline
    \texttt{modelOutput/}            & \texttt{model/output/}                  & model output: chemical species, photolysis rates, environment variables, diagnostic variables.\\
    \hline
    \texttt{instantaneousRates/}     & \texttt{model/output/reactionRates/}    & model output: reaction rates of every reaction.\\
    \hline
    \texttt{obj/}                    & \texttt{obj/}                           & files generated by the Fortran compiler.\\
    \hline
    \texttt{src/}                    & \texttt{src/}                           & Fortran source files.\\
    \hline
    \texttt{tools/}                  & \texttt{tools/}                         & various scripts and plotting tools.\\
    \hline
    \texttt{travis/}                 & \texttt{travis/}                        & scripts and files for the Test Suite.\\
  \end{tabular}
\end{table}

In AtChem2 v1.1 (and later versions), the \texttt{build/},
\texttt{mcm/}, \texttt{obj/} and \texttt{src/} directories contain the
source code, some MCM data files, the compilation files and scripts
(see Sect.~\ref{sec:build} for detailed information), while the
\texttt{travis/} directory contains the \hyperref[sec:test-suite]{Test Suite}.

For the majority of users the most important directories are
\texttt{doc/}, which contains this manual and other documentation,
\texttt{tools/}, which contains the installation and plotting scripts
(plus other utilities), and \texttt{model/}, which contains the model
information (configuration, input, output) and, usually, the chemical
mechanism. For more detailed information on these three directories,
see the following sections.

\subsection{The doc/ and tools/ directories} \label{subsec:doc-tools-directories}

The \texttt{doc/} directory contains the pdf file of the Atchem2 user
manual (this document: \texttt{AtChem2-Manual.pdf}), along with the
directory of corresponding \LaTeX\ files and figures. An electronic
copy of the poster presented at the 2018 ACM conference
\citep{sommariva_2018} is also included.

The \texttt{tools/} directory contains some auxiliary scripts (e.g.,
\texttt{version.sh} which is used to update the version number of the
model in each \hyperref[ch:development]{development} cycle) and the
following subdirectories:

\begin{itemize}
\item \texttt{install/}, which contains the the example
  \texttt{Makefile} and the scripts to install the
  \hyperref[sec:dependencies]{Dependencies}.
\item \texttt{plot/}, which contains plotting scripts in various
  programming languages.
\end{itemize}

The plotting scripts in \texttt{tools/plot/} are very simple and they
are only intended to give the user a quick overview of the model
output for validation and diagnostic purposes. We suggest to use a
proper data analysis software (e.g., IDL, Igor, MATLAB, Origin, R,
etc\ldots) to process and analyze the model results.

The plotting scripts are all similar, but they are written in
different programming languages: gnuplot, Octave~\footnote{Octave is
  an open source implementation of MATLAB. The script
  \texttt{plot-atchem2.m} is compatible with both Octave and MATLAB},
Python (v2 and v3), R. One or more of these environments should
already be installed on your system: check the local documentation or
ask the system administrator. All plotting scripts require one
argument -- the directory with the model output -- and create a file
called \texttt{atchem2\_output.pdf} in the given directory (typically
\texttt{model/output/}, but see Sect.~\ref{subsec:model-directory} for
more information).

To run a plotting script, open the terminal and execute one of the
following commands from the \maindir:

\begin{verbatim}
gnuplot -c tools/plot/plot-atchem2.gp model/output/
octave tools/plot/plot-atchem2.m model/output/
python2 tools/plot/plot-atchem2_v2.py model/output/
python3 tools/plot/plot-atchem2_v3.py model/output/
Rscript --vanilla tools/plot/plot-atchem2.r model/output/
\end{verbatim}

\subsection{The model/ directory} \label{subsec:model-directory}

The \texttt{model/} directory is the most important for the user: it
includes the model configuration files, the model constraints and the
model output. Basically, all the information required to set up an
AtChem2 model, together with the results from its run, is contained in
this directory. In principle, the chemical mechanism (\texttt{.fac}
file) could be located in another directory, although it is good
practice to keep it together with the rest of the model configuration
(see Sect.~\ref{subsec:build-process}).

The \texttt{model/} directory can be given any name and can be located
outside of the \maindir. Moreover, there can be multiple
\texttt{model/} directories (with different names) in the same
location. The paths to the required \texttt{model/} directory and/or
chemical mechanism file are given as an argument to the build script
and to the \texttt{atchem2} executable, as explained in
Sect.~\ref{sec:build}.

This approach gives the user the flexibility to run different versions
of the same model (in terms of configuration and/or chemical
mechanism) or different models (e.g., for separate projects) at the
same time, without having to recompile the source code and create a
different executable each time. Sensitivity studies and batch model
runs are therefore easy to do, since all the parts of the model than
need to be modified are contained in the same directory (i.e.,
\texttt{model/}). For more information go to Chapt.~\ref{ch:execution}.
