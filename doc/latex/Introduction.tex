\chapter{Introduction} \label{ch:introduction}

\textbf{AtChem2} is a modelling software designed to build and run
atmospheric chemistry box-models using the Master Chemical Mechanism
(\href{http://mcm.leeds.ac.uk/MCM/}{MCM}). It can also be used with
other chemical mechanisms, as long as they are provided in the FACSIMILE
format (see \{{[}\}\{{[}\}2.1 Chemical Mechanism\{{]}\}\{{]}\}).

AtChem2 was developed from \textbf{AtChem-online}
(https://atchem.leeds.ac.uk/webapp/) with the objective to create a
software able to run large atmospheric chemistry models. AtChem-online
is a web tool developed at the University of Leeds as part of the
\href{https://www.eurochamp.org/}{EUROCHAMP project}. It was designed to
facilitate the use of the MCM in the simulation of environmental chamber
experiments. A tutorial for AtChem-online, with examples and exercises,
is available on the
\href{http://mcm.leeds.ac.uk/MCMv3.3.1/atchem/tutorial_intro.htt}{MCM
website}. A help page with detailed instructions and description of the
model parameters and variables is available
\href{https://atchem.leeds.ac.uk/webapp/run/help.html}{here}.

The latest stable version of Atchem2 can be found at the
\href{https://github.com/AtChem/AtChem2/releases}{releases page}. The
development version can be downloaded from the
\href{https://github.com/AtChem/AtChem2/archive/master.zip}{master
branch} or obtained via \textbf{git}. To install and run AtChem2 follow
the instructions on the \{{[}\}\{{[}\}installation\textbar{}1.
Installation\{{]}\}\{{]}\} and \{{[}\}\{{[}\}dependencies\textbar{}1.1
Dependencies\{{]}\}\{{]}\} pages.

AtChem2 is open source, under \textbf{MIT license}. For instructions on
how to cite the model in publications, see the \texttt{CITATION.md}
file. The contributors and funders of \textbf{AtChem-online} and
\textbf{AtChem2} are listed in the \{{[}\}\{{[}\}credits
page\textbar{}Acknowledgements and Credits\{{]}\}\{{]}\}.

Bug reports, suggestions and contributions are welcome. In order to
contribute to the model development, please follow the instructions in
the corresponding \{{[}\}\{{[}\}wiki page\textbar{}3. Model
Development\{{]}\}\{{]}\}.
