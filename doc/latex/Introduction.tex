% -----------------------------------------------------------------------------
%
% Copyright (c) 2017 Sam Cox, Roberto Sommariva
%
% This file is part of the AtChem2 software package.
%
% This file is covered by the MIT license which can be found in the file
% LICENSE.md at the top level of the AtChem2 distribution.
%
% -----------------------------------------------------------------------------

\chapter{Introduction} \label{ch:introduction}

\textbf{AtChem2} is an open source modelling tool for atmospheric chemistry.
Is is designed to build and run zero-dimensional box-models using the
Master Chemical Mechanism (\href{http://mcm.leeds.ac.uk/MCM/}{MCM}).
The \textbf{MCM} is a near-explicit chemical mechanism which describes
the gas-phase oxidation of primary emitted Volatile Organic Compounds
(VOC) to carbon dioxide (\cf{CO2}) and water (\cf{H2O}). The MCM
protocol is detailed in \citet{jenkin_1997} and subsequent updates
\citep{saunders_2003, jenkin_2003, bloss_2005, jenkin_2015}. Although
it is meant to be used with the MCM, AtChem2 can easily be adapted to
use any other chemical mechanism, as long as it is provided in the
correct format.

AtChem2 is a development of \textbf{AtChem-online}, a modelling tool
created to facilitate the use of the MCM in the simulation of
laboratory and environmental chamber experiments within the
\href{https://www.eurochamp.org/}{EUROCHAMP} community
\citep{martin_2009}. AtChem-online runs as a web service provided by
the University of Leeds and can be accessed at
\href{https://atchem.leeds.ac.uk/webapp/}{https://atchem.leeds.ac.uk/webapp/}:
it can be used with only a text editor, file compression software, and a web
browser. A tutorial -- with examples and exercises -- is available on the MCM
\href{http://mcm.leeds.ac.uk/MCMv3.3.1/atchem/tutorial_intro.htt}{website}.

AtChem-online is easy to use even for inexperienced users but has a
number of limitations, mostly related to its nature as a web
application. AtChem2 runs offline and is capable of running long
simulations of computationally intensive models, as well as batch
simulations for sensitivity studies, two tasks that are not possibile
with AtChem-online. In addition, AtChem2 implements a continuous
integration workflow, coupled with a comprehensive suite of tests and
version control software (\href{https://git-scm.com/}{git}), which
makes the codebase robust, reliable and traceable.

AtChem2 is open source -- released under \textbf{MIT license} -- and is hosted
at \href{https://github.com/AtChem/AtChem2}{https://github.com/AtChem/AtChem2}.
This document (\texttt{AtChem2-Manual.pdf}) is the AtChem2 user manual
and contains all the information required to install, set up and use AtChem2.
A summary of the instructions, and additional information, can
be found on the \href{https://github.com/AtChem/AtChem2/wiki/}{wiki}.

% -------------------------------------------------------------------- %
\section{How to cite} \label{sec:how-to-cite}

AtChem2 is free to use, compatible with the terms of the MIT license;
a copy of the license can be found in the \texttt{LICENSE.md} file.

If the model is used in a publication, please include a citation to
the following paper:\\

R.~Sommariva, S.~Cox, C.~Martin, K.~Boro{\'n}ska, J.~Young, P.~K. Jimack,
M.~J. Pilling, V.~N. Matthaios, B.~S. Nelson, M.~J. Newland, M.~Panagi,
W.~J. Bloss, P.~S. Monks, and A.~R. Rickard.
\textbf{AtChem (version 1), an open-source box model for the Master Chemical Mechanism}.
\textit{Geoscientific Model Development}, 13, 1, 169--183, 2020.
doi: \href{https://doi.org/10.5194/gmd-13-169-2020}{10.5194/gmd-13-169-2020}.
