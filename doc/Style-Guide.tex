\chapter{Style Guide}

In order to make the code more readable, we attempt to use a consistent
style of coding. Two scripts, \texttt{tools/fix\_style.py} and
\texttt{tools/fix\_indent.py}, help with keeping the style of the
Fortran code consistent:

\begin{itemize}
\item
  \texttt{tools/fix\_style.py} edits files in-place to try to be
  consistent with the style guide (passing two arguments sends the
  output to the second argument, leaving the input file untouched, and
  is thus the safer option). This script is by no means infallible.;
  therefore, when using the script (by invoking
  \texttt{python\ tools/fix\_style.py\ filename}), it is strongly
  recommended to have a backup of the file to revert to, in case this
  script wrongly edits.\\
  This script is also used in the {[}{[}test suite\textbar{}3.1 Test
  Suite{]}{]} to check a few aspects of the styling. This works by
  running the script over the source file and outputting to a
  \texttt{.cmp} file: if the copy matches the original file, then the
  test passes.
\item
  \texttt{tools/fix\_indent.py} works similarly, but checks and corrects
  the indentation level of each line of code. This is also used within
  the {[}{[}test suite\textbar{}3.1 Test Suite{]}{]}.
\end{itemize}

\begin{center}\rule{0.5\linewidth}{\linethickness}\end{center}

\hypertarget{style-recommendations}{%
\subsection{Style recommendations}\label{style-recommendations}}

\hypertarget{general-principles}{%
\subsubsection{General principles}\label{general-principles}}

\begin{itemize}
\tightlist
\item
  All code should be within a module structure, except the main program.
  In our case, due to a complicating factor with linking to CVODE, we
  also place \texttt{FCVFUN()} and \texttt{FCVJTIMES()} within the main
  file \texttt{atchem.f90}.
\item
  Code is write in free-form Fortran, so source files should end in
  \texttt{.f90}
\item
  Use two spaces to indent blocks
\item
  Comment each procedure with a high-level explanation of what that
  procedure does.
\item
  Comment at the top of each file with author, date, purpose of code.
\item
  Anything in comments is not touched by the style guide, although
  common sense rules, and any code within comments should probably
  follow the rules below.
\end{itemize}

\hypertarget{specific-recommendations}{%
\subsubsection{Specific
recommendations}\label{specific-recommendations}}

\begin{itemize}
\tightlist
\item
  All \textbf{keywords} are lowercase, e.g.~\texttt{if\ then},
  \texttt{call}, \texttt{module}, \texttt{integer}, \texttt{real},
  \texttt{only}, \texttt{intrinsic}. This also includes
  \texttt{(kind=XX)} and \texttt{(len=XX)} statements.
\item
  All \textbf{intrinsic} function names are lowercase,
  e.g.~\texttt{trim}, `adjustl', 'adjustr`.
\item
  \textbf{Relational operators} should use \texttt{\textgreater{}=},
  \texttt{==} rather than \texttt{.GE.}, \texttt{.EQ.}, and surrounded
  by a single space.
\item
  \texttt{=} should be surrounded by one space when used as assignment,
  except in the cases of \texttt{(kind=XX)} and \texttt{(len=XX)} where
  no spaces should be used.
\item
  \textbf{Mathematical operators} should be surrounded by one space,
  e.g.~\texttt{*}, \texttt{-}, \texttt{+}, \texttt{**}.

  \begin{itemize}
  \tightlist
  \item
    The case of scientific number notation requires no spaces around the
    \texttt{+} or \texttt{-}, e.g.~\texttt{1.5e-9}.
  \end{itemize}
\item
  \textbf{Variables} begin with lowercase, while \textbf{procedures}
  (that is, subroutines and functions) begin with uppercase. An
  exception is \textbf{third-party functions}, which should be
  uppercase. Use either CamelCase or underscores to write multiple-word
  identifiers.
\item
  \textbf{All procedures and modules} should include the `implicit none'
  statement.
\item
  All variable \textbf{declarations} should include the \texttt{::}
  notation.
\item
  All procedure dummy arguments should include an \textbf{intent}
  statement in their declaration.
\item
  \textbf{Brackets}:

  \begin{itemize}
  \tightlist
  \item
    Opening brackets always have no space before them, except for
    \texttt{read}, \texttt{write}, \texttt{open}, \texttt{close}
    statements.
  \item
    \texttt{call} statements, and the definitions of all procedures
    should contain \textbf{one} space inside the brackets before the
    first argument and after the last argument,
    e.g.~\texttt{call\ function\_name(\ arg1,\ arg2\ )},
    \texttt{subroutine\ subroutine\_name(\ arg1\ )}
  \item
    Functions calls, and array indices have \textbf{no such space}
    before the first argument or after the last argument.
  \end{itemize}
\end{itemize}
