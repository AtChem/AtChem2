\section{Photolysis rates} \label{sec:photolysis}

The photolysis rates are identified in {[}{[}FACSIMILE
format\textbar{}2.1 Chemical Mechanism{]}{]} as
\texttt{J\textless{}n\textgreater{}}, where \texttt{n} is an integer
determined by the
\href{http://mcm.leeds.ac.uk/MCMv3.3.1/parameters/photolysis.htt}{MCM
naming convention}. The photolysis rates are calculated by AtChem2 using
the
\href{http://mcm.leeds.ac.uk/MCM/parameters/photolysis_param.htt}{MCM
parametrization}, as explained in more detail below. Each photolysis
rate can also be set to a constant value or to constrained values.

The following rules apply:

\begin{enumerate}
\def\labelenumi{\arabic{enumi}.}
\item
  If a photolysis rate is set as constant, it assumes the given value.
  Any other photolysis rate, without an explicitly defined constant
  value, is set to zero.
\item
  If one or more photolysis rates are set to constrained (and none is
  set to constant), they assume the values given in the corresponding
  constraint files. Any other photolysis rate is calculated.
\item
  If no photolysis rate is set to constant or to constrained, the model
  calculates all the photolysis rates.
\end{enumerate}

The environment variable \texttt{ROOF} can also be used to turn the
photolysis rates ON/OFF, which is useful for simulations of some
environmental chamber experiments (see {[}{[}2.4 Environment
Variables{]}{]}).

\hypertarget{constant-photolysis-rates}{%
\subsection{Constant photolysis rates}\label{constant-photolysis-rates}}

The typical scenario for constant photolysis rates is the use of a lamp
in an environmental chamber. All the photolysis rates used in the
mechanism need to be given a value (in
\texttt{model/configuration/photolysisConstant.config}) otherwise they
will be set to zero. This approach allows the user to model individual
photolysis processes and/or to account for lamps that emit only in
certain spectral windows. The format of the
\texttt{photolysisConstant.config} file is described in the
{[}{[}configuration files page\textbar{}2.6 Config Files{]}{]}.

\hypertarget{constrained-photolysis-rates}{%
\subsection{Constrained photolysis
rates}\label{constrained-photolysis-rates}}

Photolysis rates can be constrained to measured values. In this case,
the name of the constrained photolysis rate (e.g., \texttt{J2}) must be
in \texttt{model/configuration/photolysisConstrained.config} and a file
with the constraint data must be present in
\texttt{model/constraints/photolysis/}. For more information go to:
{[}{[}2.6 Config Files{]}{]} and {[}{[}2.7 Constraints{]}{]}.

It is not always possibile to measure - and therefore constrain - all
the required photolysis rates. The photolysis rates that are not
constrained (i.e., not listed in \texttt{photolysisConstrained.config})
are calculated using the MCM parametrization.

\hypertarget{calculated-photolysis-rates}{%
\subsection{Calculated photolysis
rates}\label{calculated-photolysis-rates}}

AtChem2 implements the parametrization of photolysis rates used by the
Master Chemical Mechanism. It is described in the MCM protocol papers:
\href{https://doi.org/10.1016/S1352-2310(96)00105-7}{Jenkin et al.,
Atmos. Environ., 31, 81, 1997} and
\href{https://doi.org/10.5194/acp-3-161-2003}{Saunders et al., Atmos.
Chem. Phys., 3, 161, 2003}.

The MCM parametrization calculates the photolysis rate of a reaction
(\texttt{J}) with the equation:

\begin{verbatim}
J = l * (cosX)^m * exp(-n * secX) * tau
\end{verbatim}

where \texttt{l}, \texttt{m}, \texttt{n} are empirical parameters,
\texttt{cosX} is the cosine of the solar zenith angle, \texttt{secX} is
the inverse of \texttt{cosX} (i.e., \texttt{secX\ =\ 1/cosX}) and
\texttt{tau} is the transmission factor. The empirical parameters are
different for each version of the MCM. AtChem2 v1.1 includes the
empircal parameters for
\href{http://mcm.leeds.ac.uk/MCM/parameters/photolysis_param.htt}{version
3.3.1} in the file \texttt{mcm/photolysis-rates\_v3.3.1}. This file also
contains the transmission factor \texttt{tau}, which can be changed by
the user (by default \texttt{tau\ =\ 1}). It is possible to use previous
versions of the MCM parametrization: see the file \texttt{mcm/INFO.md}
for instructions.

The solar zenith angle is calculated by AtChem2 using latitude,
longitude, time of the day and sun declination (see {[}{[}2.2 Model
Parameters{]}{]} and {[}{[}2.4 Environment Variables{]}{]}). The
calculation is detailed in ``The Atmosphere and UV-B Radiation at Ground
Level'' (\href{https://doi.org//10.1007/978-1-4899-2406-3_1}{S.
Madronich, Environmental UV Photobiology, 1993}).

\hypertarget{jfac}{%
\subsection{JFAC}\label{jfac}}

Measurements of ambient photolysis rates typically show short-term
variability, due to the changing meteorological conditions (clouds,
rain, etc\ldots{}). This information is retained in the constrained
photolysis rates, but it is lost in the calculated ones. To account for
this, the calculated photolysis rates can be scaled by a correction
factor (\texttt{JFAC}), as explained below.

The environment variable \texttt{JFAC} is a constant or time-dependent
parameter that can be used to correct the calculated photolysis rates
for external factors not taken into account by the MCM parametrization,
such as cloudiness. \texttt{JFAC} is defined as the ratio between a
measured and the calculated photolysis rate. Typically \texttt{J4} (the
photolysis rate of NO2) is used for this purpose, as it is one of the
most frequently measured photolysis rates.

\begin{verbatim}
JFAC = j(NO2)/J4
\end{verbatim}

where \texttt{j(NO2)} is the measured value and \texttt{J4} is
calculated with the MCM parametrization (see above). \texttt{JFAC} is by
default 1, meaning that the calculated photolyis rates are not scaled;
it can be set to any value between 0 and 1 (see {[}{[}2.4 Environment
Variables{]}{]}) or it can be constrained (see {[}{[}2.7
Constraints{]}{]}). Note that only the photolysis rates calculated with
the MCM parameterization are scaled by \texttt{JFAC}, the constrained
and the constant photolysis rates are not.

\texttt{JFAC} can also be calculated at runtime. To do so, \texttt{JFAC}
should be set to the name of the photolysis rate to be used as reference
(e.g., \texttt{J4}) in
\texttt{model/configuration/environmentVariables.config}. There should
be an associated constraint file in
\texttt{model/constraints/environment/}. \textbf{Important}: this option
is not working very well in the current version of AtChem2, so it is
suggested to calculate \texttt{JFAC} offline and to constrain it (see
issue \href{https://github.com/AtChem/AtChem2/issues/16}{\#16}).
