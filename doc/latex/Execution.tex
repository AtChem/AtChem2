% -----------------------------------------------------------------------------
%
% Copyright (c) 2017 Sam Cox, Roberto Sommariva
%
% This file is part of the AtChem2 software package.
%
% This file is covered by the MIT license which can be found in the file
% LICENSE.md at the top level of the AtChem2 distribution.
%
% -----------------------------------------------------------------------------

\chapter{Model Execution} \label{ch:execution}

% -------------------------------------------------------------------- %
\section{The Model} \label{sec:model}

AtChem2 is designed to build and run atmospheric chemistry box-models
based upon the Master Chemical Mechanism (\href{http://mcm.leeds.ac.uk/MCM/}{MCM}).
Other chemical mechanisms can be used, as long as they are provided in
the correct format, as explained in Sect.~\ref{sec:chemical-mechanism}.

This chapter explains how to set up, compile and run an AtChem2
atmospheric chemistry model. The directory structure of AtChem2 is
described in Sect.~\ref{sec:model-structure}.
A working knowledge of the \textbf{unix shell} and its
\href{https://swcarpentry.github.io/shell-novice/reference/}{basic commands}
is \emph{required} to use the AtChem2 model.

There are two sets of inputs to AtChem2 -- the mechanism file, and the
configuration files.

\subsection{Mechanism file} \label{subsec:mechanism-file}

AtChem2 requires a chemical mechanism in
\hyperref[subsec:facsimile-format]{FACSIMILE format} (\texttt{.fac}).
The mechanism file can be downloaded from the MCM website
 using the \href{http://mcm.leeds.ac.uk/MCM/extract.htt}{extraction tool}
or it can be assembled manually. The user can modify the downloaded
\texttt{.fac} file with a text editor.

The mechanism file is converted into a shared library (\texttt{.so})
and a number of related files during the build process (see
Sect.~\ref{subsec:build-process}).

\subsection{Configuration files} \label{subsec:configuration-files}

The model configuration is set via a number of text files located in
the \texttt{model/configuration/} directory:

\begin{itemize}
\item model and solver parameters -- go to
  \hyperref[sec:model-parameters]{Model Parameters} and
  \hyperref[sec:solver-parameters]{Solver Parameters}.
\item environment variables settings -- go to
  \hyperref[sec:environment-variables]{Environment Variables}.
\item photolysis rates settings -- go to
  \hyperref[sec:photolysis-rates]{Photolysis Rates}.
\item initialization of chemical species, model output settings -- go
  to \hyperref[sec:config-files]{Config Files}.
\end{itemize}

All the configuration files can be modified with a text
editor. Detailed information on the configuration files can be found
in the \hyperref[ch:setup]{Model Setup} chapter. The model constraints
-- chemical species, environment variables, photolysis rates -- are
located in the \texttt{model/constraints/} directory: for more
information go to Sect.~\ref{sec:constraints}.

% -------------------------------------------------------------------- %
\section{Constraints} \label{sec:constraints}

AtChem2 can be run in two modes:

\begin{itemize}
\item unconstrained: all variables are calculated by the model from
  the initial conditions, which are set in the model
  \hyperref[sec:config-files]{configuration files}.
\item constrained: one or more variables are constrained, meaning that
  the solver forces their value to a given value at each time
  step. The variables that are not constrained are calculated by the
  model.
\end{itemize}

The constrained values must be provided as one file for each
constrained variable, with the format described below. By default, the
files with the constraint data are in \texttt{model/constraints/species/}
for the chemical species, \texttt{model/constraints/environment/} for
the environment variables, and \texttt{model/constraints/photolysis/}
for the photolysis rates. The default directories can be changed, as
explained in Sect.~\ref{subsec:model-directory}.

\subsection{Constrained variables} \label{subsec:constrained-variables}

\subsubsection{Environment variables}

All environment variables, except \texttt{ROOF}, can be
constrained. To do so, set the variable to \texttt{CONSTRAINED} in
\texttt{model/configuration/environmentVariables.config} and create
the file with the constraint data. The name of the file must be the
same as the name of the variable, e.g., \texttt{TEMP} (without
extension). See also Sect.~\ref{sec:environment-variables}.

\subsubsection{Chemical species}

Any chemical species in the chemical mechanism can be constrained. To
do so, add the name of the species to
\texttt{model/configuration/speciesConstrained.config} and create the
file with the constraint data. The name of the file must be the same
as the name of the chemical species, e.g., \texttt{CH3OH} (without
extension). See also Sect.~\ref{sec:config-files}.

\subsubsection{Photolysis rates}

Any of the photolysis rates in the chemical mechanism can be
constrained. The photolysis rates are identified as \verb|J<n>|, where
\texttt{n} is an integer (Sect.~\ref{sec:photolysis-rates}). To
constrain a photolysis rate add its name (e.g., \texttt{J4}) to
\texttt{model/configuration/photolysisConstrained.config} and create
the file with the constraint data. The name of the file must be the
same as the name of the photolysis rate, e.g., \texttt{J4} (without
extension). See also Sect.~\ref{sec:config-files}.

\subsection{Constraint files} \label{subsec:constraint-files}

The files with the constraint data are text files with two columns
separated by spaces. The first column is the time in \textbf{seconds}
from midnight of day/month/year (see Sect.~\ref{sec:model-parameters}),
the second column is the value of the variable in the appropriate
unit. For the chemical species the unit is \textbf{molecules cm-3} and
for the photolysis rates the unit is \textbf{s-1}; for the environment
variables see Sect.~\ref{sec:environment-variables}. For example:

\begin{verbatim}
-900   73.21
0      74.393
900    72.973
1800   72.63
2700   72.73
3600   69.326
4500   65.822
5400   63.83
6300   64.852
7200   64.739
\end{verbatim}

The time in the first column of a constraint file can be negative.
AtChem2 interprets negative times as ``seconds \emph{before} midnight
of day/month/year'' (see Sect.~\ref{sec:model-parameters}). This can
be useful to allow correct interpolation of the variables at the
beginning of the model run, as explained below.

The model constraints \emph{must} cover the same amount of time, or
preferably more, as the intended model runtime. For example: if the
model starts at 42300 seconds and stops at 216000 seconds, the first
and the last data points in a constraint file must have a time of
42300 seconds (or lower) and 21600 seconds (or higher), respectively.

\subsection{Interpolation} \label{subsec:interpolation}

Constraints can be provided at different timescales. Typically, the
constraint data come from direct measurements and it is very common
for different instruments to sample with different frequencies. For
example, ozone and nitrogen oxides can be measured once every minute,
but most organic compounds can be measured only once every hour.

The user can average the constraints so that they are all at the same
timescale or can use the data with the original timestamps. Both
approaches have advantages and disadvantages in terms of how much
pre-processing work is required, and in terms of model accuracy and
integration speed \citep{sommariva_2019}. Whether all the constraints
have the same timescale or not, the solver interpolates between data
points using the interpolation method selected in
\texttt{model/configuration/model.parameters}
(Sect.~\ref{sec:model-parameters}). The default interpolation method
is piecewise linear, but piecewise constant interpolation is also
available.

The photolysis rates and the environment variables are evaluated by
the solver when needed -- each is interpolated individually, only when
constrained. This happens each time the function
\texttt{mechanism\_rates()} is called from \texttt{FCVFUN()}, and is
controlled by \textbf{CVODE} as it carries out the integration. In a
similar way, the interpolation routine for the chemical species is
called once for each of the constrained species in \texttt{FCVFUN()},
plus once when setting the initial conditions of each of the
constrained species.

As mentioned above, the model start and stop time \emph{must be}
within the time interval of the constrained data to avoid
interpolation errors or model crash. If data is not supplied for the
full runtime interval, then the \emph{final} value will be used for
all times both \emph{before the first data point} and \emph{after the
  last data point}. This behaviour is likely to change in future
versions, at least to avoid the situation where the last value is used
for all times before the first (see issue
\href{https://github.com/AtChem/AtChem2/issues/294}{\#294}). A warning
is printed for all evaluations outside of the supplied time
interval. It is good practice to supply data that cover a short time
\emph{beyond} the final model time, which may be used by the solver.

% -------------------------------------------------------------------- %
\section{Build} \label{sec:build}

The script \texttt{build\_atchem2.sh} in the \texttt{build/} directory
is used to process the chemical mechanism file (\texttt{.fac}) and to
compile the model. The script generates one Fortran file
(\texttt{mechanism.f90}), one shared library (\texttt{mechanism.so})
and four mechanism files (\texttt{mechanism.prod},
\texttt{mechanism.reac}, \texttt{mechanism.ro2},
\texttt{mechanism.species}) in the \texttt{model/configuration/}
directory.

The \texttt{build/build\_atchem2.sh} script must be run from the
\maindir\ and takes four arguments which must be passed in the exact
order indicated below. This means that if -- for example -- the third
argument needs to be specified, it is also necessary to specify the
first and the second arguments, even if they have the default
values. To avoid mistakes, the user can choose to always specify all
the arguments. The four arguments, and their default values, are:

\begin{enumerate}
\item the path to the chemical mechanism file -- no default
  (suggested: \texttt{model/}).
\item the path to the directory for the Fortran and mechanism files
  generated from the \texttt{.fac} file -- default:\\
  \texttt{model/configuration/}.
\item the path to the directory containing the configuration files --
  default:\\
  \texttt{model/configuration/}.
\item the path to the directory containing the MCM data files -- default:\\
  \texttt{mcm/}.
\end{enumerate}

For example, if the chemical mechanism file is in the \texttt{model/}
directory, the model is build using the command:

\begin{verbatim}
./build/build_atchem2.sh model/mechanism.fac model/configuration/
                         model/configuration/ mcm/
\end{verbatim}

An installation of AtChem2 can have multiple \texttt{model/}
directories, which may correspond to different models or different
projects; this allows the user to run more than one model at the same
time. In the following example, the \maindir\ contains two
\texttt{model/} directories with different names (\texttt{model\_1}
and \texttt{model\_2}, each with their own chemical mechanism,
configuration, constraints and output:

\begin{verbatim}
AtChem2/
        | mcm/
        | model_1/
             | configuration/
             | constraints/
             | output/
             | mechanism.fac
        | model_2/
             | configuration/
             | constraints/
             | output/
             | mechanism.fac
        | obj/
        | src/
        | tools/
        | travis/
\end{verbatim}

The \texttt{model/} directories can also be located outside the
\maindir: as long as the correct paths are passed to the
\texttt{build\_atchem2.sh} script and to the exectuable (see below),
the model will compile and run. For example:

\begin{verbatim}
./build/build_atchem2.sh model_1/mechanism.fac model_1/configuration/
                         model_1/configuration/
./build/build_atchem2.sh model_2/mechanism.fac model_2/configuration/
                         model_2/configuration/
\end{verbatim}

Note that the fourth argument is not specified in the example above,
so the default value (\texttt{mcm/}) will be used.

Compilation is required only once for a given \texttt{.fac} file. If
the user changes the configuration files, there is no need to
recompile the model. Likewise, if the constraints files are changed,
there is no need to recompile the model. This is because the model
configuration and the model constraints are read by the executable at
runtime. However, if the user makes changes to the \texttt{.fac} file,
then the shared library \texttt{model/configuration/mechanism.so}
needs to be recompiled using the \texttt{build\_atchem2.sh} script.

The user may also want, or need, to change the Fortran code
(\texttt{src/*.f90}), in which case the model needs to be recompiled:
if the \texttt{.fac} file has also been changed, use the
\texttt{build\_atchem2.sh} script. Otherwise, if only the Fortran code
has been changed, executing \texttt{make} from the \maindir\ is enough
to recompile the model.

% -------------------------------------------------------------------- %
\section{Execute} \label{sec:execute}

The compilation process creates an executable file called
\texttt{atchem2} in the \maindir. The executable file
takes seven arguments, corresponding to the directories containing the
model configuration and output:

\begin{enumerate}
\item the path to the directory for the model output -- default:\\
  \texttt{model/output}
\item the path to the directory for the model output reaction rates -
  default:\\
  \texttt{model/output/reactionRates/}
\item the path to the directory with the configuration files --
  default:\\
  \texttt{model/configuration/}.
\item the path to the directory with the MCM data files -- default:\\
  \texttt{mcm/}.
\item the path to the directory with the data files of constrained
  chemical species -- default:\\
  \texttt{model/constraints/species/}
\item the path to the directory with the data files of constrained
  environment variables -- default:\\
  \texttt{model/constraints/environment/}
\item the path to the directory with the data files of constrained
  photolysis rates -- default:\\
  \texttt{model/constraints/photolysis/}
\end{enumerate}

The model can be run simply by executing the \texttt{atchem2} command
from the \maindir, in which case the executable will use the default
configuration and output directories. Otherwise, the configuration and
output directories need to be specified. AtChem2 uses the following
flags to pass the arguments to the executable: \texttt{--model},
\texttt{--output}, \texttt{--reactionRates}, \texttt{--configuration},
\texttt{--constraints}, \texttt{--env\_constraints},
\texttt{--photo\_constraints}, \texttt{--spec\_constraints},
\texttt{--mcm}, and \texttt{--shared-lib}.

The command \texttt{atchem2 --help} displays a help message showing
the usage of the command line arguments. For example, if the
constraints are in the default directories (or not used), the model
can be run by executing:

\begin{verbatim}
./atchem2 --output=model/output/
          --reactionRates=model/output/reactionRates/
          --configuration=model/configuration/
          --spec_constraints=model_1/constraints/species/
          --env_constraints=model_1/constraints/environment/
          --photo_constraints=model_1/constraints/photolysis/
          --mcm=mcm/
\end{verbatim}

In the case of multiple \texttt{model/} directories, the directories
corresponding to each model need to be passed as arguments to the
\texttt{atchem2} executable. This allows the user to run two or more
models simultaneously. For example:

\begin{verbatim}
    ./atchem2 --output=model_1/output/
              --configuration=model_1/configuration/
              --constraints=model_1/constraints/
    ./atchem2 --output=model_2/output/
              --configuration=model_2/configuration/
              --constraints=model_2/constraints/
\end{verbatim}

As explained above, if the chemical mechanism (\texttt{.fac}) is
changed, only the shared library needs to be recompiled. This allows
the user to have only one base executable called \texttt{atchem2} in
the \maindir: when running multiple models at the same
time the user can reuse this base executable while pointing each model
to the correct shared library and configuration files.

% -------------------------------------------------------------------- %
\section{Output} \label{sec:output}

The model output is saved by default in the directory
\texttt{model/output/}. The location can be modified by changing the
arguments of the \texttt{atchem2} executable, as explained in
Sect.~\ref{sec:execute}.

The AtChem2 output files are space-delimited text files, with a header
containing the names of the variables:

\begin{itemize}
\item values of environment variables and concentrations of chemical
  species:\\
  \texttt{environmentVariables.output},
  \texttt{speciesConcentrations.output}.
\item values of photolysis rates and related parameters:\\
  \texttt{photolysisRates.output},
  \texttt{photolysisRatesParameters.output}.
\item loss and production rates of selected species (see
  Sect.~\ref{sec:config-files} for details):\\
  \texttt{lossRates.output},
  \texttt{productionRates.output}.
\item Jacobian matrix (if requested, see Sect.~\ref{sec:model-parameters}):\\
  \texttt{jacobian.output}.
\item model diagnostic variables:\\
  \texttt{errors.output}
  \texttt{finalModelState.output},
  \texttt{mainSolverParameters.output}.
\end{itemize}

In addition, the reaction rates of all the reactions in the chemical
mechanism are saved in the directory \texttt{reactionRates/}: one file
for each model step, with the filename corresponding to the time in
seconds. While the model is running diagnostic information is printed
to the terminal: this can be redirected to a log file using standard
unix commands. A successfull model run completes with a message
similar to the one shown in Sect.~\ref{sec:install}.
