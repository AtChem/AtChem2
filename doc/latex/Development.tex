% -----------------------------------------------------------------------------
%
% Copyright (c) 2017 Sam Cox, Roberto Sommariva
%
% This file is part of the AtChem2 software package.
%
% This file is covered by the MIT license which can be found in the file
% LICENSE.md at the top level of the AtChem2 distribution.
%
% -----------------------------------------------------------------------------

\chapter{Model Development} \label{ch:development}

% -------------------------------------------------------------------- %
\section{General Information} \label{sec:general-information}

There are two versions of AtChem2 in the main
\href{https://github.com/AtChem/AtChem2}{github repository}:

\begin{enumerate}
\item Stable version: is indicated by a version number (e.g.,
  \textbf{v1.0}), and can be downloaded from the
  \href{https://github.com/AtChem/AtChem2/releases}{Releases page}.
\item Development version -- i.e. the \texttt{master\ branch}: is
  indicated by a version number with the suffix \texttt{-dev} (e.g.,
  \textbf{v1.1-dev}), and can be downloaded as an
  \href{https://github.com/AtChem/AtChem2/archive/master.zip}{archive
    file} or obtained via \textbf{git} (Sect.~\ref{sec:download}).
\end{enumerate}

AtChem2 is under active development, which means that the
\texttt{master\ branch} is sometimes a few steps ahead of the latest
stable release. Any modification to the code is automatically run
through the \hyperref[sec:test-suite]{Testsuite} before it is merged
into the \texttt{master\ branch}. The Testsuite is designed to ensure
that changes to the code do not cause unintended behaviour or
unexplained differences in the model results, so the development
version is normally safe to use. However, it is recommended to use the
stable version for ``production runs'' and publications since it can
be more easily referenced -- see the discussion about traceability and
reproducibility of computational models in \citet{sommariva_2020}.

Feedback, bug reports, comments and suggestions are welcome: the list
of open issues and known bugs can be found on the related
\href{https://github.com/AtChem/AtChem2/issues}{github page}. The
preferred way to contribute to the development of AtChem2 is to use
\textbf{git}: instructions on how to set up git and submit
contributions to the AtChem2 repository can be found on the
\href{https://github.com/AtChem/AtChem2/wiki/How-to-contribute}{wiki}.
The coding guidelines for Fortran are detailed in Sect.\ref{sec:style-guide}.

% -------------------------------------------------------------------- %
\section{Testsuite} \label{sec:test-suite}

AtChem2 uses \href{https://github.com/features/actions}{GitHub Actions} for
\textbf{continuous integration} testing. This programming approach
ensures that changes to the code do not modify the behaviour and the
results of the software in an unintended fashion. The Testsuite is in
the \texttt{tests/} directory and consists of a series of tests and
short model runs that check the model functionality and calculations
against known outputs. The \href{https://codecov.io}{Codecov} service
is used to verify that the Testsuite covers a significant fraction of
the codebase ($>$90\%) and a wide range of common configurations.

There are four types of tests, which can be executed from the
\maindir\ using the command \verb|make| (note that the
\hyperref[subsec:optional-dependencies]{optional dependencies} need to
be installed):

\begin{itemize}
\item \textbf{Indent} and \textbf{Style} tests: check that the
  indentation and coding style of the Fortran code are consistent with
  the guidelines -- \verb|make indenttest| and \verb|make styletest|.
\item \textbf{Unit} tests: check that individual functions generate
  the expected outputs -- \verb|make unittests|.
\item \textbf{Behaviour} tests: build and run a number of models with
  different configurations and check that they generate the expected
  outputs -- \verb|make modeltests|.
\end{itemize}

The command \verb|make alltests| runs all the tests in the Testsuite
in succession. Each test outputs the results to the terminal and, in
case of failure, a log file (\texttt{tests/testsuite.log}) is
generated for the indent, style and behaviour tests. If all tests are
successfully passed the following message is printed to the terminal:

\begin{verbatim}
Style test       PASSED
Indent test      PASSED
Tests            PASSED
  (20/20 tests PASSED)
Testsuite PASSED
\end{verbatim}

The Testsuite is automatically run every time a Pull Request (PR) is created
or updated on the main github repository (\texttt{AtChem/AtChem2}).  The Pull
Request triggers a build on GitHub Actions which runs the entire Testsuite on
two architectures (Linux and macOS) using different versions of the GNU
\texttt{gfortran} compiler. The CI tester performs the following tasks on each
architecture:

\begin{itemize}
\item Install \texttt{gfortran}, CVODE, and \texttt{numdiff}:
  \begin{itemize}
  \item Linux: use \texttt{apt-get} to install \texttt{gfortran}. Install
    CVODE from source~\footnote{\texttt{apt-get} could also be used to install
      SUNDIALS, but the repository does not currently hold CVODE v2.9.}.
  \item macOS: use \texttt{Homebrew} to install \texttt{gfortran}. Install
    CVODE from source.
  \end{itemize}
\item Install the other dependencies (\texttt{openlibm}, \texttt{numdiff} and
  FRUIT) using the scripts in the \texttt{tools/install/} directory.
\item Build and run the example AtChem2 model using the default
  configuration. PASS if it exits with 0.
\item Build and run the indent and style tests. PASS if all tests pass.
\item Build and run the unit tests. PASS if all unit tests pass.
\item Build and run the behaviour tests. PASS if no differences from
  the reference output files are found, otherwise FAIL.
\end{itemize}

Every test must pass to allow the full CI to pass. This is indicated
by the message ``All checks have passed'' on the github PR page. Pull
Requests should only be merged into the \texttt{master\ branch} once
GitHub Actions has completed with passes on both architectures.

\subsection{Adding new unit tests} \label{subsec:adding-new-unit-tests}

The unit tests are in the \texttt{tests/unit\_tests/} directory and
require the FRUIT optional dependency (which requires Ruby,
Sect.~\ref{subsec:optional-dependencies}). To add new unit tests,
follow the procedure outlined below:

\begin{itemize}
\item The unit test files are called \texttt{*\_test.f90}. If the new
  unit test to be added fits into an existing test file, edit that
  file -- otherwise, create a new test file following the same naming
  pattern. It is suggested that unit tests covering functions from the
  source file \texttt{xFunctions.f90} should be named
  \texttt{x\_test.f90}.
\item The unit test file must contain a module with the same name as
  the file -- i.e. \texttt{x\_test} -- and it must include the
  statement \verb|use fruit|, plus any other required module.
\item The module should contain a number of subroutines with the
  naming pattern \texttt{test\_*}. These subroutines must take no
  arguments and, importantly, must not have any brackets for arguments
  -- subroutine \texttt{test\_calc} is correct, but subroutine
  \texttt{test\_calc()} is wrong.
\item Each subroutine should call one or more assert functions:
  usually, these are \texttt{assert\_equals()}, \texttt{assert\_not\_equals()},
  \texttt{assert\_true()}, \texttt{assert\_false()}. The assert functions
  act as the arbiters of pass or failure of the unit test -- each
  assert must pass for the subroutine to pass, and each subroutine
  must pass for the unit tests to pass.
\item The assert functions have the following syntax:
  \begin{verbatim}
  call assert_true( a == b , "Test that a and b are equal")
  call assert_false( a == b , "Test that a and b are not equal")
  call assert_equals( a, b , "Test that a and b are equal")
  call assert_not_equals( a, b , "Test that a and b are not equal")
  \end{verbatim}
\end{itemize}

It is useful to use the last argument of the assert function as a
\emph{unique} and \emph{descriptive} message. When a unit test fails,
it is highlighted in the FRUIT output summary, and the message of the
assert function is printed. Unique and descriptive messages thus
enable faster and easier understanding of which test has failed, and
perhaps why.

If these steps are followed, calling \texttt{make\ unittests} is
enough to run all the unit tests, including the new ones. To verify
that the new tests have indeed been run and passed, check the output
summary -- there should be a line associated to each of the
\texttt{test\_*} subroutines in each test file.

\subsection{Adding new behaviour tests} \label{subsec:adding-new-behaviour-tests}

Each behaviour test (\texttt{\$TESTNAME}) is contained in its own
subdirectory inside the \texttt{tests/tests/} directory. A behaviour
test requires the following files and directory structure:

\begin{verbatim}
|- $TESTNAME
|  |- configuration
|  |  |- environmentVariables.config
|  |  |- initialConcentrations.config
|  |  |- mechanism.prod.cmp
|  |  |- mechanism.reac.cmp
|  |  |- mechanism.ro2.cmp
|  |  |- mechanism.species.cmp
|  |  |- model.parameters
|  |  |- outputRates.config
|  |  |- outputSpecies.config
|  |  |- photolysisConstant.config     [*]
|  |  |- photolysisConstrained.config  [*]
|  |  |- solver.parameters
|  |  |- speciesConstant.config        [*]
|  |  |- speciesConstrained.config     [*]
|  |- constraints      [**]
|     |- environment/  [**]
|     |- photolysis/   [**]
|     |- species/      [**]
|- output
|  |- reactionRates/
|  |- environmentVariables.output.cmp
|  |- errors.output.cmp
|  |- finalModelState.output.cmp
|  |- jacobian.output.cmp
|  |- lossRates.output.cmp
|  |- mainSolverParameters.output.cmp
|  |- photolysisRates.output.cmp
|  |- photolysisRatesParameters.output.cmp
|  |- productionRates.output.cmp
|  |- speciesConcentrations.output.cmp
|- $TESTNAME.fac
|- $TESTNAME.out.cmp
\end{verbatim}

The files marked with \texttt{[*]} and the directories marked with
\texttt{[**]} are optional -- depending on the configuration used in
the test. If present, the directories marked with \texttt{[**]} should
contain the relevant constraint files, according to the corresponding
configuration files in \texttt{model/configuration/} (see
Sect.~\ref{sec:constraints} for details).

The file \texttt{\$TESTNAME.out.cmp} should contain the exact copy of the
expected terminal printout. Each behaviour test is briefly described in the
\texttt{tests/tests/INFO.md} file, which should be updated after a new test is
added. New tests added to the \texttt{tests/tests/} directory are
automatically picked up by the \texttt{Makefile} when running
\verb|make modeltests| or \verb|make alltests| from the \maindir.

% -------------------------------------------------------------------- %
\section{Style Guide} \label{sec:style-guide}

In order to make the AtChem2 code more readable and easier to
maintain, the source code should follow a consistent style
(Sect.~\ref{subsec:style-recommendations}). Two Python scripts are used
to check and correct the Fortran code:

\begin{itemize}
\item \texttt{fix\_style.py} edits a Fortran file in-place to make the
  code consistent with the style recommendations.
\item \texttt{fix\_indent.py} works in a similar way, but only looks
  at the indentation level of each line of code.
\end{itemize}

These scripts are in the \texttt{tools/} directory and can be invoked
from the \maindir\ with the following commands:

\begin{verbatim}
python tools/fix_style.py src/filename.f90
python tools/fix_indent.py src/filename.f90
\end{verbatim}

It is important to keep in mind that these scripts are \emph{not
  infallible} and, therefore, it is strongly recommended to always
have a backup of the Fortran file to revert to, in case of wrong
edits. This can be done by passing two arguments to the script instead
of one: the second argument sends the script output to another file,
leaving the original file untouched.

Both scripts are also used in the Testsuite to run the style and
indent tests (Sec.~\ref{sec:test-suite}): each script is run over each
source file and the output is sent to a \texttt{.cmp} file. If the
\texttt{.cmp} file matches the original file, the test passes.

\subsection{Style recommendations} \label{subsec:style-recommendations}

\subsubsection{General principles}

\begin{itemize}
\item All code should be organized in a \textbf{module structure},
  except the main program. There is only one exception: due to a
  complicating factor with linking to CVODE, the functions
  \texttt{FCVFUN()} and \texttt{FCVJTIMES()} are placed within the
  main file \texttt{atchem.f90}.
\item All code should be written in free-form Fortran, and the source
  files should have the extension \texttt{.f90}.
\item Always use two spaces to indent blocks.
\item At the top of each file there should be a header indicating the
  author(s), date, and purpose of the code; if necessary,
  acknowledgements to other contributors should be added.
\item Always comment a procedure with a high-level explanation of what
  that procedure does.
\item There are no specific guidelines for comments, although common
  sense applies and any code within the comments should broadly follow
  the rules below.
\end{itemize}

\subsubsection{Specific recommendations}

\begin{itemize}
\item All \textbf{keywords} should be lowercase, e.g., \texttt{if\
    then}, \texttt{call}, \texttt{module}, \texttt{integer},
  \texttt{real}, \texttt{only}, \texttt{intrinsic}. This includes the
  \texttt{(kind=XX)} and \texttt{(len=XX)} statements.
\item All \textbf{intrinsic} function names should be lowercase, e.g.,
  \texttt{trim}, \texttt{adjustl}, \texttt{adjustr}.
\item The \textbf{relational operators} should use \texttt{$\geq$} and
  \texttt{==} rather than \texttt{.GE.}, \texttt{.EQ.}, and should be
  surrounded by a single space.
\item The \texttt{=} operator should be surrounded by one space when
  used as assignment -- except in the cases of the \texttt{(kind=XX)}
  and \texttt{(len=XX)} statements, where no spaces should be used.
\item The \textbf{mathematical operators} (\texttt{*}, \texttt{-},
  \texttt{+}, \texttt{**}) should be surrounded by one space.
\item Numbers in scientific notation should have no spaces around the
  \texttt{+} or \texttt{-}, e.g., \texttt{1.5e-9}.
\item The names of \textbf{variables} should begin with lowercase,
  while those of \textbf{procedures} (that is, subroutines and
  functions) should begin with uppercase. An exception is
  \textbf{third-party functions}, which should be uppercase. Use
  either CamelCase or underscores to write multiple-word identifiers.
\item All \textbf{modules} should include the \texttt{implicit none}
  statement.
\item All \textbf{variable declarations} should include the
  \texttt{::} notation.
\item The \textbf{dummy arguments} of a procedure should include an
  \texttt{intent} statement in their declaration.
\item The following rules apply to \textbf{brackets}:
  \begin{itemize}
  \item Opening brackets should not have a space before them, except
    for \texttt{read}, \texttt{write}, \texttt{open}, \texttt{close}
    statements.
  \item All \texttt{call} statements and the definitions of all
    procedures should contain one space before the first argument and
    one after the last argument inside the brackets:
    \begin{verbatim}
    call function_name( arg1, arg2 )
    subroutine subroutine_name( arg1 )
    \end{verbatim}
  \item Functions calls and array indices should not have spaces
    before the first argument or after the last argument inside the
    brackets.
  \end{itemize}
\end{itemize}
